\documentclass[12pt]{article}
\usepackage{pmmeta}
\pmcanonicalname{DirectProductOfAlgebras}
\pmcreated{2013-03-22 16:44:35}
\pmmodified{2013-03-22 16:44:35}
\pmowner{CWoo}{3771}
\pmmodifier{CWoo}{3771}
\pmtitle{direct product of algebras}
\pmrecord{9}{38967}
\pmprivacy{1}
\pmauthor{CWoo}{3771}
\pmtype{Definition}
\pmcomment{trigger rebuild}
\pmclassification{msc}{08A05}
\pmclassification{msc}{08A62}
\pmdefines{direct product}
\pmdefines{direct factor}
\pmdefines{direct power}
\pmdefines{projection}
\pmdefines{empty direct product}

\endmetadata

\usepackage{amssymb,amscd}
\usepackage{amsmath}
\usepackage{amsfonts}

% used for TeXing text within eps files
%\usepackage{psfrag}
% need this for including graphics (\includegraphics)
%\usepackage{graphicx}
% for neatly defining theorems and propositions
\usepackage{amsthm}
% making logically defined graphics
%%\usepackage{xypic}
\usepackage{pst-plot}
\usepackage{psfrag}

% define commands here
\newtheorem{prop}{Proposition}
\newtheorem{thm}{Theorem}
\newtheorem{ex}{Example}
\newcommand{\real}{\mathbb{R}}
\begin{document}
In this entry, let $O$ be a fixed operator set.  All algebraic systems have the same type (they are all $O$-algebras).

Let $\lbrace A_i\mid i\in I\rbrace$ be a set of algebraic systems of the same type ($O$) indexed by $I$.  Let us form the Cartesian product of the underlying sets and call it $A$:
$$A:=\prod_{i\in I} A_i.$$
Recall that element $a$ of $A$ is a function from $I$ to $\bigcup A_i$ such that for each $i\in I$, $a(i)\in A_i$.  

For each $\omega\in O$ with arity $n$, let $\omega_{A_i}$ be the corresponding $n$-ary operator on $A_i$.  Define $\omega_A: A^n\to A$ by 
$$\omega_A(a_1,\ldots,a_n)(i)=\omega_{A_i}(a_1(i),\ldots,a_n(i))\quad\mbox{ for all }i\in I.$$
One readily checks that $\omega_A$ is a well-defined $n$-ary operator on $A$.  $A$ equipped with all $\omega_A$ on $A$ is an $O$-algebra, and is called the \emph{direct product} of $A_i$.  Each $A_i$ is called a \emph{direct factor} of $A$.

If each $A_i=B$, where $B$ is an $O$-algebra, then we call $A$ the direct power of $B$ and we write $A$ as $B^I$ (keep in mind the isomorphic identifications).

If $A$ is the direct product of $A_i$, then for each $i\in I$ we can associate a  homomorphism $\pi_i:A\to A_i$ called a \emph{projection} given by $\pi_i(a)=a(i)$.  It is a homomorphism because $\pi_i(\omega_A(a_1,\ldots, a_n))=\omega_A(a_1,\ldots, a_n)(i)=\omega_{A_i}(a_1(i),\ldots,a_n(i))=\omega_{A_i}(\pi_i(a_1),\ldots, \pi_i(a_n))$.

\textbf{Remark}.  The direct product of a single algebraic system is the algebraic system itself.  An \emph{empty direct product} is defined to be a trivial algebraic system (one-element algebra).
%%%%%
%%%%%
\end{document}
