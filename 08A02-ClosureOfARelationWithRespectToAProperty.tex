\documentclass[12pt]{article}
\usepackage{pmmeta}
\pmcanonicalname{ClosureOfARelationWithRespectToAProperty}
\pmcreated{2013-03-22 17:36:26}
\pmmodified{2013-03-22 17:36:26}
\pmowner{CWoo}{3771}
\pmmodifier{CWoo}{3771}
\pmtitle{closure of a relation with respect to a property}
\pmrecord{15}{40023}
\pmprivacy{1}
\pmauthor{CWoo}{3771}
\pmtype{Definition}
\pmcomment{trigger rebuild}
\pmclassification{msc}{08A02}
\pmrelated{Property2}
\pmdefines{reflexive closure}
\pmdefines{symmetric closure}
\pmdefines{transitive closure}
\pmdefines{reflexive transitive closure}

\usepackage{amssymb,amscd}
\usepackage{amsmath}
\usepackage{amsfonts}
\usepackage{mathrsfs}

% used for TeXing text within eps files
%\usepackage{psfrag}
% need this for including graphics (\includegraphics)
%\usepackage{graphicx}
% for neatly defining theorems and propositions
\usepackage{amsthm}
% making logically defined graphics
%%\usepackage{xypic}
\usepackage{pst-plot}
\usepackage{psfrag}

% define commands here
\newtheorem{prop}{Proposition}
\newtheorem{thm}{Theorem}
\newtheorem{ex}{Example}
\newcommand{\real}{\mathbb{R}}
\newcommand{\pdiff}[2]{\frac{\partial #1}{\partial #2}}
\newcommand{\mpdiff}[3]{\frac{\partial^#1 #2}{\partial #3^#1}}
\begin{document}
\PMlinkescapeword{proposition}
\PMlinkescapeword{words}
\PMlinkescapeword{closure}
\PMlinkescapeword{parent}

\subsubsection*{Introduction}

Fix a set $A$.  A \emph{property} $\mathcal{P}$ of $n$-ary relations on a set $A$ may be thought of as some subset of the set of all $n$-ary relations on $A$.  Since an $n$-ary relation is just a subset of $A^n$, $\mathcal{P}\subseteq P(A^n)$, the powerset of $A^n$.  An $n$-ary relation is said to have property $\mathcal{P}$ if $R \in \mathcal{P}$.

For example, the transitive property is a property of binary relations on $A$; it consists of all transitive binary relations on $A$.  Reflexive and symmetric properties are sets of reflexive and symmetric binary relations on $A$ correspondingly.

Let $R$ be an $n$-ary relation on $A$.  By the \emph{closure} of an $n$-ary relation $R$ with respect to property $\mathcal{P}$, or the $\mathcal{P}$-closure of $R$ for short, we mean the smallest relation $S\in \mathcal{P}$ such that $R\subseteq S$.  In other words, if $T\in \mathcal{P}$ and $R\subseteq T$, then $S\subseteq T$.  We write $\operatorname{Cl}_{\mathcal{P}}(R)$ for the $\mathcal{P}$-closure of $R$.  

Given an $n$-ary relation $R$ on $A$, and a property $\mathcal{P}$ on $n$-ary relations on $A$, does $\operatorname{Cl}_{\mathcal{P}}(R)$ always exist?  The answer is no.  For example, let $\mathcal{P}$ be the anti-symmetric property of binary relations on $A$, and $R=A^2$.  For another example, take $\mathcal{P}$ to be the irreflexive property, and $R=\Delta$, the diagonal relation on $A$.

However, if $A^n \in \mathcal{P}$ and $\mathcal{P}$ is closed under arbitrary intersections, then $\mathcal{P}$ is a complete lattice according to \PMlinkname{this fact}{CriteriaForAPosetToBeACompleteLattice}, and, as a result, $\operatorname{Cl}_{\mathcal{P}}(R)$ exists for any $R\subseteq A^n$.

\subsubsection*{Reflexive, Symmetric, and Transitive Closures}

From now on, we concentrate on binary relations on a set $A$.  In particular, we fix a binary relation $R$ on $A$, and let $\mathcal{X}$ the reflexive property, $\mathcal{S}$ the symmetric property, and $\mathcal{T}$ be the transitive property on the binary relations on $A$.

\begin{prop} Arbitrary intersections are closed in $\mathcal{X}$, $\mathcal{S}$, and $\mathcal{T}$.  Furthermore, if $R$ is any binary relation on $A$, then 
\begin{itemize}
\item $R^=:=\operatorname{Cl}_{\mathcal{X}}(R)=R\cup \Delta$, where $\Delta$ is the diagonal relation on $A$,
\item $R^{\leftrightarrow}:=\operatorname{Cl}_{\mathcal{S}}(R)=R\cup R^{-1}$, where $R^{-1}$ is the converse of $R$, and
\item $R^+:=\operatorname{Cl}_{\mathcal{T}}(R)$ is given by $$\bigcup_{n\in \mathbb{N}} R^n = R\cup (R\circ R) \cup \cdots \cup \underbrace{(R\circ \cdots \circ R)}_{\displaystyle{n}\mbox{-fold power}} \cup \cdots ,$$
where $\circ$ is the relational composition operator.
\item $R^*:=R^{=+}=R^{+=}$.
\end{itemize}
\end{prop}
$R^=$, $R^{\leftrightarrow}$, $R^+$, and $R^*$ are called the \emph{reflexive closure}, the \emph{symmetric closure}, the \emph{transitive closure}, and the \emph{reflexive transitive closure} of $R$ respectively.  The last item in the proposition permits us to call $R^*$ the \emph{transitive reflexive closure} of $R$ as well (there is no difference to the order of taking closures).  This is true because $\Delta$ is transitive.

\textbf{Remark}.  In general, however, the order of taking closures of a relation is important.  For example, let $A=\lbrace a,b\rbrace$, and $R=\lbrace (a,b)\rbrace$.  Then $R^{\leftrightarrow +}=A^2\ne \lbrace (a,b),(b,a)\rbrace = R^{+ \leftrightarrow}$.
%%%%%
%%%%%
\end{document}
