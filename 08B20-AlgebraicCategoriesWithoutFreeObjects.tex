\documentclass[12pt]{article}
\usepackage{pmmeta}
\pmcanonicalname{AlgebraicCategoriesWithoutFreeObjects}
\pmcreated{2013-03-22 16:51:25}
\pmmodified{2013-03-22 16:51:25}
\pmowner{Algeboy}{12884}
\pmmodifier{Algeboy}{12884}
\pmtitle{algebraic categories without free objects}
\pmrecord{4}{39104}
\pmprivacy{1}
\pmauthor{Algeboy}{12884}
\pmtype{Example}
\pmcomment{trigger rebuild}
\pmclassification{msc}{08B20}

\endmetadata

\usepackage{latexsym}
\usepackage{amssymb}
\usepackage{amsmath}
\usepackage{amsfonts}
\usepackage{amsthm}

%%\usepackage{xypic}

%-----------------------------------------------------

%       Standard theoremlike environments.

%       Stolen directly from AMSLaTeX sample

%-----------------------------------------------------

%% \theoremstyle{plain} %% This is the default

\newtheorem{thm}{Theorem}

\newtheorem{coro}[thm]{Corollary}

\newtheorem{lem}[thm]{Lemma}

\newtheorem{lemma}[thm]{Lemma}

\newtheorem{prop}[thm]{Proposition}

\newtheorem{conjecture}[thm]{Conjecture}

\newtheorem{conj}[thm]{Conjecture}

\newtheorem{defn}[thm]{Definition}

\newtheorem{remark}[thm]{Remark}

\newtheorem{ex}[thm]{Example}



%\countstyle[equation]{thm}



%--------------------------------------------------

%       Item references.

%--------------------------------------------------


\newcommand{\exref}[1]{Example-\ref{#1}}

\newcommand{\thmref}[1]{Theorem-\ref{#1}}

\newcommand{\defref}[1]{Definition-\ref{#1}}

\newcommand{\eqnref}[1]{(\ref{#1})}

\newcommand{\secref}[1]{Section-\ref{#1}}

\newcommand{\lemref}[1]{Lemma-\ref{#1}}

\newcommand{\propref}[1]{Prop\-o\-si\-tion-\ref{#1}}

\newcommand{\corref}[1]{Cor\-ol\-lary-\ref{#1}}

\newcommand{\figref}[1]{Fig\-ure-\ref{#1}}

\newcommand{\conjref}[1]{Conjecture-\ref{#1}}


% Normal subgroup or equal.

\providecommand{\normaleq}{\unlhd}

% Normal subgroup.

\providecommand{\normal}{\lhd}

\providecommand{\rnormal}{\rhd}
% Divides, does not divide.

\providecommand{\divides}{\mid}

\providecommand{\ndivides}{\nmid}


\providecommand{\union}{\cup}

\providecommand{\bigunion}{\bigcup}

\providecommand{\intersect}{\cap}

\providecommand{\bigintersect}{\bigcap}










\begin{document}
An initial object is always a free object.  So in the context of algebraic
systems with a trivial object, such as groups, or modules, there is always
at least one free object.  However, we usually dismiss this example as it
does not lead to any useful results such as the existence of presentations.

However, there are many ways in which a cateogry of algebraic objects can
fail to include non-trivial free objects.

\section{Restriction to finite sets}

The restriction of a category which naturally includes infinite objects
can often be restricted to just the finite objects and in so doing often
remove all non-trivial free objects.
\begin{itemize}
\item The category of finite groups has only the trivial free object.  Indeed,
even the rank 1 free group, the integers $\mathbb{Z}$ is already infinite.
\item Similarly, finite modules of in a module category over an infinite 
ring are never free. For examples use the rings $\mathbb{Z}$, $\mathbb{Z}_m[x]$,
etc.
\end{itemize}

However, this is not always the case.  For example, if we consider finite
$\mathbb{Z}_p$-modules (vector spaces) each of these are free.

\section{Homomorphism restrictions}

In the category of rings with 1 it is often beneficial to force all ring
homomorphisms to be unital.  However, this restriction can prevent
the construction of free objects.

Suppose $F$ is a free ring in the category of rings with positive characteristic.  Then we ask, what is the characteristic of $F$?

If it is $m>0$ then we choose another ring $R$ of a different characteristic,
a characteristic relatively prime to $m$, and then there can be no 
unital homomorphism from $F$ to $R$.  So $F$ must have characteristic 0.
In contrast to the above examples we have not excluded infinite objects
in this restriction.  This example is even more powerful than those
above as it also exclude the existance of an initial object, so indeed
NO free objects exist in this category.

If we return to the full category of unital rings we observe
every ring is a $\mathbb{Z}$-algebra we can use the free associative
algebras $\mathbb{Z}\langle X\rangle$ does exist here.

%%%%%
%%%%%
\end{document}
