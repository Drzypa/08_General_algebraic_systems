\documentclass[12pt]{article}
\usepackage{pmmeta}
\pmcanonicalname{PresentationsOfAlgebraicObjects}
\pmcreated{2013-03-22 16:51:27}
\pmmodified{2013-03-22 16:51:27}
\pmowner{CWoo}{3771}
\pmmodifier{CWoo}{3771}
\pmtitle{presentations of algebraic objects}
\pmrecord{5}{39105}
\pmprivacy{1}
\pmauthor{CWoo}{3771}
\pmtype{Definition}
\pmcomment{trigger rebuild}
\pmclassification{msc}{08B20}
\pmrelated{Presentationgroup}
\pmdefines{presentation}

\usepackage{latexsym}
\usepackage{amssymb}
\usepackage{amsmath}
\usepackage{amsfonts}
\usepackage{amsthm}

%%\usepackage{xypic}

%-----------------------------------------------------

%       Standard theoremlike environments.

%       Stolen directly from AMSLaTeX sample

%-----------------------------------------------------

%% \theoremstyle{plain} %% This is the default

\newtheorem{thm}{Theorem}

\newtheorem{coro}[thm]{Corollary}

\newtheorem{lem}[thm]{Lemma}

\newtheorem{lemma}[thm]{Lemma}

\newtheorem{prop}[thm]{Proposition}

\newtheorem{conjecture}[thm]{Conjecture}

\newtheorem{conj}[thm]{Conjecture}

\newtheorem{defn}[thm]{Definition}

\newtheorem{remark}[thm]{Remark}

\newtheorem{ex}[thm]{Example}



%\countstyle[equation]{thm}



%--------------------------------------------------

%       Item references.

%--------------------------------------------------


\newcommand{\exref}[1]{Example-\ref{#1}}

\newcommand{\thmref}[1]{Theorem-\ref{#1}}

\newcommand{\defref}[1]{Definition-\ref{#1}}

\newcommand{\eqnref}[1]{(\ref{#1})}

\newcommand{\secref}[1]{Section-\ref{#1}}

\newcommand{\lemref}[1]{Lemma-\ref{#1}}

\newcommand{\propref}[1]{Prop\-o\-si\-tion-\ref{#1}}

\newcommand{\corref}[1]{Cor\-ol\-lary-\ref{#1}}

\newcommand{\figref}[1]{Fig\-ure-\ref{#1}}

\newcommand{\conjref}[1]{Conjecture-\ref{#1}}


% Normal subgroup or equal.

\providecommand{\normaleq}{\unlhd}

% Normal subgroup.

\providecommand{\normal}{\lhd}

\providecommand{\rnormal}{\rhd}
% Divides, does not divide.

\providecommand{\divides}{\mid}

\providecommand{\ndivides}{\nmid}


\providecommand{\union}{\cup}

\providecommand{\bigunion}{\bigcup}

\providecommand{\intersect}{\cap}

\providecommand{\bigintersect}{\bigcap}










\begin{document}
Given an algebraic category with enough free objects one can use the general
description of the free object to provide a precise description of all other
objects in the category.  The process is called a \emph{presentation}.

Suppose $A$ is an object generated by a subset $X$.  Then if there exists
a free object on $X$, $F$, then there exists a unique morphism 
$f:F\rightarrow A$ which matches the embedding of $X$ in $F$ to the embedding 
of $X$ in $A$.

As we are in an algebraic category we have a fundamental homomorphism theorem (we take this as our definition of an algebraic category in this context).
This means there is a notion of kernel $K$ of $f$ and quotient $F/K$ such that $F/K$ is isomorphic to $A$.

Now $F$ is generted by $X$ so every element of $F$ is expressed as an informal word over $X$.  [By \emph{informal word} we mean whatever process encodes
general elements as generated by $X$.  For example, in groups and semigroups these are actual formal words, but in algebras these can be linear combinations
of words or polynomials with indeterminants in $X$, etc.]  Hence a set of generators for the kernel $K$ will be expressed as words over $X$.

\begin{defn}
A presentation of an object $A$ is a pair of sets $\langle X|R\rangle$ where 
$X$ generates $A$ and $R$ is a set of informal words over $X$ such that 
the free object $F$ on $X$ and the normal subobject $K$ of $F$ generated by $R$
has the property $F/K\cong A$.
\end{defn}

Once again, normal refers to whatever property is required for subobject to allow quotients, so normal subgroup or ideals, etc.


Existence of presentations is dependent on the category being considered.
The common categories: groups, rings, and modules all have presentations.

It is generally not possible to insist that a presentation is unique.  First we have the variable choice of generators.  Secondly, we may choose various relations.  Indeed, it is possible that the relations will generate different subobjects $K$ such that $F/K\cong A$.  In practice, presentations are a highly compactified description of an object which can hide many essential features of the object.  Indeed, in the extreem case are the theorems of Boone which show that in the category of groups it is impossible to tell if an arbitrary presentation is a presentation of the trivial group.
For a detailed account of these theorems refer to 
\begin{quote}
Joseph Rotman, \emph{An Introduction to the Theory of Groups}, Springer, New  York,  Fourth edition, 1995.
\end{quote}

%%%%%
%%%%%
\end{document}
