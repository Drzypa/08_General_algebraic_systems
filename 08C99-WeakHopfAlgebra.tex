\documentclass[12pt]{article}
\usepackage{pmmeta}
\pmcanonicalname{WeakHopfAlgebra}
\pmcreated{2013-03-22 18:12:43}
\pmmodified{2013-03-22 18:12:43}
\pmowner{bci1}{20947}
\pmmodifier{bci1}{20947}
\pmtitle{weak Hopf algebra}
\pmrecord{35}{40793}
\pmprivacy{1}
\pmauthor{bci1}{20947}
\pmtype{Definition}
\pmcomment{trigger rebuild}
\pmclassification{msc}{08C99}
\pmclassification{msc}{81R15}
\pmclassification{msc}{57T05}
\pmclassification{msc}{81R50}
\pmclassification{msc}{16W30}
\pmsynonym{quantum groupoids v.1}{WeakHopfAlgebra}
%\pmkeywords{bialgebras}
%\pmkeywords{extensions of Hopf algebras}
%\pmkeywords{Grassmann-Hopf algebras}
\pmrelated{HopfAlgebra}
\pmrelated{WeakHopfCAlgebra2}
\pmrelated{WeakHopfCAlgebra2}
\pmrelated{CommutativeDiagram}
\pmrelated{GroupoidAndGroupRepresentationsRelatedToQuantumSymmetries}
\pmrelated{GrassmanHopfAlgebrasAndTheirDualCoAlgebras}
\pmrelated{WeakHopfCAlgebra2}
\pmdefines{weak bialgebra}
\pmdefines{commutant of a set}
\pmdefines{counit axiom}
\pmdefines{antipode map}
\pmdefines{counity}
\pmdefines{twisted tensor product}
\pmdefines{quantum double}
\pmdefines{QOA}

% this is the default PlanetMath preamble.  as your 

% almost certainly you want these
\usepackage{amssymb}
\usepackage{amsmath}
\usepackage{amsfonts}

% there are many more packages, add them here as you need 

% define commands here
\usepackage{amsmath, amssymb, amsfonts, amsthm, amscd, latexsym,enumerate}
%%\usepackage{xypic}
\usepackage[mathscr]{eucal}

\setlength{\textwidth}{6.5in}
%\setlength{\textwidth}{16cm}
\setlength{\textheight}{9.0in}
%\setlength{\textheight}{24cm}

\hoffset=-.75in     %%ps format
%\hoffset=-1.0in     %%hp format
\voffset=-.4in


\theoremstyle{plain}
\newtheorem{lemma}{Lemma}[section]
\newtheorem{proposition}{Proposition}[section]
\newtheorem{theorem}{Theorem}[section]
\newtheorem{corollary}{Corollary}[section]

\theoremstyle{definition}
\newtheorem{definition}{Definition}[section]
\newtheorem{example}{Example}[section]
%\theoremstyle{remark}
\newtheorem{remark}{Remark}[section]
\newtheorem*{notation}{Notation}
\newtheorem*{claim}{Claim}

\renewcommand{\thefootnote}{\ensuremath{\fnsymbol{footnote}}}
\numberwithin{equation}{section}

\newcommand{\Ad}{{\rm Ad}}
\newcommand{\Aut}{{\rm Aut}}
\newcommand{\Cl}{{\rm Cl}}
\newcommand{\Co}{{\rm Co}}
\newcommand{\DES}{{\rm DES}}
\newcommand{\Diff}{{\rm Diff}}
\newcommand{\Dom}{{\rm Dom}}
\newcommand{\Hol}{{\rm Hol}}
\newcommand{\Mon}{{\rm Mon}}
\newcommand{\Hom}{{\rm Hom}}
\newcommand{\Ker}{{\rm Ker}}
\newcommand{\Ind}{{\rm Ind}}
\newcommand{\IM}{{\rm Im}}
\newcommand{\Is}{{\rm Is}}
\newcommand{\ID}{{\rm id}}
\newcommand{\GL}{{\rm GL}}
\newcommand{\Iso}{{\rm Iso}}
\newcommand{\rO}{{\rm O}}
\newcommand{\Sem}{{\rm Sem}}
\newcommand{\St}{{\rm St}}
\newcommand{\Sym}{{\rm Sym}}
\newcommand{\SU}{{\rm SU}}
\newcommand{\Tor}{{\rm Tor}}
\newcommand{\U}{{\rm U}}

\newcommand{\A}{\mathcal A}
\newcommand{\Ce}{\mathcal C}
\newcommand{\D}{\mathcal D}
\newcommand{\E}{\mathcal E}
\newcommand{\F}{\mathcal F}
\newcommand{\G}{\mathcal G}
\renewcommand{\H}{\mathcal H}
\renewcommand{\cL}{\mathcal L}
\newcommand{\Q}{\mathcal Q}
\newcommand{\R}{\mathcal R}
\newcommand{\cS}{\mathcal S}
\newcommand{\cU}{\mathcal U}
\newcommand{\W}{\mathcal W}

\newcommand{\bA}{\mathbb{A}}
\newcommand{\bB}{\mathbb{B}}
\newcommand{\bC}{\mathbb{C}}
\newcommand{\bD}{\mathbb{D}}
\newcommand{\bE}{\mathbb{E}}
\newcommand{\bF}{\mathbb{F}}
\newcommand{\bG}{\mathbb{G}}
\newcommand{\bK}{\mathbb{K}}
\newcommand{\bM}{\mathbb{M}}
\newcommand{\bN}{\mathbb{N}}
\newcommand{\bO}{\mathbb{O}}
\newcommand{\bP}{\mathbb{P}}
\newcommand{\bR}{\mathbb{R}}
\newcommand{\bV}{\mathbb{V}}
\newcommand{\bZ}{\mathbb{Z}}

\newcommand{\bfE}{\mathbf{E}}
\newcommand{\bfX}{\mathbf{X}}
\newcommand{\bfY}{\mathbf{Y}}
\newcommand{\bfZ}{\mathbf{Z}}

\renewcommand{\O}{\Omega}
\renewcommand{\o}{\omega}
\newcommand{\vp}{\varphi}
\newcommand{\vep}{\varepsilon}

\newcommand{\diag}{{\rm diag}}
\newcommand{\grp}{{\mathsf{G}}}
\newcommand{\dgrp}{{\mathsf{D}}}
\newcommand{\desp}{{\mathsf{D}^{\rm{es}}}}
\newcommand{\Geod}{{\rm Geod}}
\newcommand{\geod}{{\rm geod}}
\newcommand{\hgr}{{\mathsf{H}}}
\newcommand{\mgr}{{\mathsf{M}}}
\newcommand{\ob}{{\rm Ob}}
\newcommand{\obg}{{\rm Ob(\mathsf{G)}}}
\newcommand{\obgp}{{\rm Ob(\mathsf{G}')}}
\newcommand{\obh}{{\rm Ob(\mathsf{H})}}
\newcommand{\Osmooth}{{\Omega^{\infty}(X,*)}}
\newcommand{\ghomotop}{{\rho_2^{\square}}}
\newcommand{\gcalp}{{\mathsf{G}(\mathcal P)}}

\newcommand{\rf}{{R_{\mathcal F}}}
\newcommand{\glob}{{\rm glob}}
\newcommand{\loc}{{\rm loc}}
\newcommand{\TOP}{{\rm TOP}}

\newcommand{\wti}{\widetilde}
\newcommand{\what}{\widehat}

\renewcommand{\a}{\alpha}
\newcommand{\be}{\beta}
\newcommand{\ga}{\gamma}
\newcommand{\Ga}{\Gamma}
\newcommand{\de}{\delta}
\newcommand{\del}{\partial}
\newcommand{\ka}{\kappa}
\newcommand{\si}{\sigma}
\newcommand{\ta}{\tau}

\newcommand{\med}{\medbreak}
\newcommand{\medn}{\medbreak \noindent}
\newcommand{\bign}{\bigbreak \noindent}

\newcommand{\lra}{{\longrightarrow}}
\newcommand{\ra}{{\rightarrow}}
\newcommand{\rat}{{\rightarrowtail}}
\newcommand{\ovset}[1]{\overset {#1}{\ra}}
\newcommand{\ovsetl}[1]{\overset {#1}{\lra}}
\newcommand{\hr}{{\hookrightarrow}}
%\usepackage{geometry, amsmath,amssymb,latexsym,enumerate}
%%%\usepackage{xypic}

\def\baselinestretch{1.1}


\hyphenation{prod-ucts}

%\geometry{textwidth= 16 cm, textheight=21 cm}

\newcommand{\sqdiagram}[9]{$$ \diagram  #1  \rto^{#2} \dto_{#4}&
#3  \dto^{#5} \\ #6    \rto_{#7}  &  #8   \enddiagram
\eqno{\mbox{#9}}$$ }

\def\C{C^{\ast}}

\newcommand{\labto}[1]{\stackrel{#1}{\longrightarrow}}

%\newenvironment{proof}{\noindent {\bf Proof} }{ \hfill $\Box$
%{\mbox{}}

\newcommand{\quadr}[4]{\begin{pmatrix} & #1& \\[-1.1ex] #2 & & #3\\[-1.1ex]& #4&
 \end{pmatrix}}
\def\D{\mathsf{D}}
\begin{document}
\textbf{Definition 0.1}: 
In order to define a \emph{weak Hopf algebra}, one weakens, or relaxes certain axioms of a Hopf algebra as follows~: 

\begin{itemize}
\item[(1)]
The comultiplication is not necessarily unit--preserving.

\item[(2)]
The counit $\vep$ is not necessarily a homomorphism of algebras.

\item[(3)]
The axioms for the antipode map $S : A \lra A$ with respect to the
counit are as follows. For all $h \in H$,
\begin{equation}
\begin{aligned} m(\ID \otimes S) \Delta (h) &= (\vep \otimes
\ID)(\Delta (1) (h \otimes 1)) \\ m(S \otimes \ID) \Delta (h) &=
(\ID \otimes \vep)((1 \otimes h) \Delta(1)) \\ S(h) &= S(h_{(1)})
h_{(2)}  S(h_{(3)}) ~.
\end{aligned}
\end{equation}
\end{itemize}

These axioms may be appended by the following commutative diagrams
\begin{equation}
{\begin{CD} A \otimes A @> S\otimes \ID >> A \otimes A
\\ @A \Delta AA   @VV m V
 \\ A @ > u \circ \vep >> A
\end{CD}} \qquad
{\begin{CD} A \otimes A @> \ID\otimes S >> A \otimes A
\\ @A \Delta AA   @VV m V
 \\ A @ > u \circ \vep >> A
\end{CD}}
\end{equation}
along with the counit axiom:
\begin{equation}
\xymatrix@C=3pc@R=3pc{ A \otimes A \ar[d]_{\vep \otimes 1} & A
\ar[l]_{\Delta} \ar[dl]_{\ID_A} \ar[d]^{\Delta}
\\ A  & A \otimes A \ar[l]^{1 \otimes \vep}}
\end{equation}

Some authors substitute the term \emph{quantum groupoid} for a weak Hopf algebra.
Therefore, the weak Hopf algebra is considered by some authors as an important
concept in quantum operator algebra (QOA).


\subsection{Examples of weak Hopf algebras}
\begin{itemize}
\item [(1)]
We refer here to Bais et al. (2002). Let $G$ be a non-Abelian group
and $H \subset G$ a discrete subgroup. Let $F(H)$ denote the space
of functions on $H$ and $\bC H$ the group algebra (which consists
of the linear span of group elements with the group structure).

The \emph{quantum double} $D(H)$ (Drinfeld, 1987) is defined by
\begin{equation}
D(H) = F(H)~ \wti{\otimes}~ \bC H~,
\end{equation}
where, for $x \in H$, the {\em twisted tensor product} is specified by
\begin{equation}
\wti{\otimes} \mapsto ~(f_1 \otimes h_1) (f_2 \otimes h_2)(x) =
f_1(x) f_2(h_1 x h_1^{-1}) \otimes h_1 h_2 ~.
\end{equation}
The physical interpretation is often to take $H$ as the `electric gauge group' and $F(H)$ as the `magnetic symmetry' generated by $\{f \otimes e\}$~. In terms of the counit $\vep$, the double
$D(H)$ has a trivial representation given by $\vep(f \otimes h) =
f(e)$~. We next look at certain features of this construction.

\med
For the purpose of braiding relations there is an $R$ matrix, $R
\in D(H) \otimes D(H)$, leading to the operator
\begin{equation}
\mathcal R \equiv \sigma \cdot (\Pi^A_{\a} \otimes \Pi^B_{\be})
(R)~,
\end{equation}
in terms of the Clebsch--Gordan series $\Pi^A_{\a} \otimes
\Pi^B_{\be} \cong N^{AB \gamma}_{\a \be C}~ \Pi^C_{\gamma}$, and
where $\sigma$ denotes a flip operator. The operator $\mathcal
R^2$ is sometimes called the \emph{monodromy} or
\emph{Aharanov--Bohm phase factor}. In the case of a condensate in
a state $\vert v \rangle$ in the carrier space of some
representation $\Pi^A_{\a}$~. One considers the maximal Hopf
subalgebra $T$ of a Hopf algebra $A$ for which $\vert v \rangle$
is $T$--invariant; specifically ~:
\begin{equation}
\Pi^A_{\a} (P)~\vert v \rangle = \vep(P) \vert v \rangle~,~
\forall P \in T~.
\end{equation}

\med
\item[(2)]
For the second example, consider $A = F(H)$~. The algebra of
functions on $H$ can be broken to the algebra of functions on
$H/K$, that is, to $F(H/K)$, where $K$ is normal in $H$, that is,
$HKH^{-1} =K$~. Next, consider $A = D(H)$~. On breaking a purely
electric condensate $\vert v \rangle$, the magnetic symmetry
remains unbroken, but the electric symmetry $\bC H$ is broken to
$\bC N_v$, with $N_v \subset H$, the stabilizer of $\vert v
\rangle$~. From this we obtain $T = F(H) \wti{\otimes} \bC N_v$~.

\med
\item[(3)]
In Nikshych and Vainerman (2000) quantum groupoids (as weak
C*--Hopf algebras, see below) were studied in relationship to the
noncommutative symmetries of depth 2 von Neumann subfactors. If
\begin{equation}
A \subset B \subset B_1 \subset B_2 \subset \ldots
\end{equation}
is the Jones extension induced by a finite index depth $2$
inclusion $A \subset B$ of $II_1$ factors, then $Q= A' \cap B_2$
admits a quantum groupoid structure and acts on $B_1$, so that $B
= B_1^{Q}$ and $B_2 = B_1 \rtimes Q$~. Similarly, in Rehren (1997)
`paragroups' (derived from weak C*--Hopf algebras) comprise
(quantum) groupoids of equivalence classes such as associated with
6j--symmetry groups (relative to a fusion rules algebra). They
correspond to type $II$ von Neumann algebras in quantum mechanics,
and arise as symmetries where the local subfactors (in the sense
of containment of observables within fields) have depth 2 in the
Jones extension. Related is how a von Neumann algebra $N$, such as
of finite index depth 2, sits inside a weak Hopf algebra formed as
the crossed product $N \rtimes A$ (B\"ohm et al. 1999).

\med
\item[(4)]
In Mack and Schomerus (1992) using a more general notion of the
Drinfeld construction, develop the notion of a \emph{quasi
triangular quasi--Hopf algebra} (QTQHA) is developed with the aim
of studying a range of essential symmetries with special
properties, such the quantum group algebra $\U_q (\rm{sl}_2)$ with
$\vert q \vert =1$~. If $q^p=1$, then it is shown that a QTQHA is
canonically associated with $\U_q (\rm{sl}_2)$. Such QTQHAs are
claimed as the true symmetries of minimal conformal field
theories.
\end{itemize}


\section{Definitions of Related Concepts}
Let us recall two basic concepts of quantum operator algebra that are essential to Algebraic Quantum Theories. \\

\subsection {Definition of a Von Neumann Algebra.}
\med
Let $\H$ denote a complex (separable) Hilbert space. A \emph{von
Neumann algebra} $\A$ acting on $\H$ is a subset of the algebra of
all bounded operators $\cL(\H)$ such that:

\begin{itemize}

\item[(1)] $\A$ is closed under the adjoint operation (with the
adjoint of an element $T$ denoted by $T^*$).

\med
\item[(2)]
$\A$ equals its bicommutant, namely:

\begin{equation}
\A= \{A \in \cL(\H) : \forall B \in \cL(\H), \forall C\in \A,~
(BC=CB)\Rightarrow (AB=BA)\}~.
\end{equation}
\end{itemize}

If one calls a \emph{commutant} of a set $\A$ the special set of
bounded operators on $\cL(\H)$ which commute with all elements in
$\A$, then this second condition implies that the commutant of the
commutant of $\A$ is again the set $\A$.

\med
On the other hand, a von Neumann algebra $\A$ inherits a
\emph{unital} subalgebra from $\cL(\H)$, and according to the
first condition in its definition $\A$ does indeed inherit a
\emph{*-subalgebra} structure, as further explained in the next
section on C*-algebras. Furthermore, we have notable
\emph{Bicommutant Theorem} which states that $\A$ \emph{is a von
Neumann algebra if and only if $\A$ is a *-subalgebra of
$\cL(\H)$, closed for the smallest topology defined by continuous
maps $(\xi,\eta)\longmapsto (A\xi,\eta)$ for all $<A\xi,\eta)>$
where $<.,.>$ denotes the inner product defined on $\H$}~. For
further instruction on this subject, see e.g. Aflsen and Schultz
(2003), Connes (1994).


\subsection{Definition of a Hopf algebra}
\med
Firstly, a {\em unital associative algebra} consists of a linear space
$A$ together with two linear maps

\begin{equation}
\begin{aligned} m &: A \otimes A \lra A~,~(multiplication) \\
\eta &: \bC \lra A~,~ (unity)
\end{aligned}
\end{equation}
satisfying the conditions
\begin{equation}
\begin{aligned}
m(m \otimes \mathbf 1) &= m (\mathbf 1 \otimes m)  \\  m(\mathbf 1
\otimes \eta) &= m (\eta \otimes \mathbf 1) = \ID~.
\end{aligned}
\end{equation}
This first condition can be seen in terms of a commuting diagram~:
\begin{equation}
\begin{CD}
A \otimes A \otimes A @> m \otimes \ID>> A \otimes A
\\ @V \ID \otimes mVV   @VV m V
 \\ A \otimes A  @ > m >> A
\end{CD}
\end{equation}
Next suppose we consider `reversing the arrows', and take an
algebra $A$ equipped with a linear homorphisms $\Delta : A \lra A
\otimes A$, satisfying, for $a,b \in A$ :

\begin{equation}
\begin{aligned} \Delta(ab) &= \Delta(a) \Delta(b)
\\ (\Delta \otimes \ID) \Delta &= (\ID \otimes \Delta) \Delta~.
\end{aligned}
\end{equation}

We call $\Delta$ a \emph{comultiplication}, which is said to be
\emph{coasociative} in so far that the following diagram commutes
\begin{equation}
\begin{CD}
A \otimes A \otimes A @< \Delta\otimes \ID<< A \otimes A
\\ @A \ID \otimes \Delta AA  @AA \Delta A
 \\ A \otimes A  @ < \Delta << A
\end{CD}
\end{equation}

There is also a counterpart to $\eta$, the \emph{counity} map
$\vep : A \lra \bC$ satisfying
\begin{equation}
(\ID \otimes \vep) \circ \Delta = (\vep \otimes \ID) \circ \Delta
= \ID~.
\end{equation}
A \emph{bialgebra} $(A, m, \Delta, \eta,
\vep)$ is a linear space $A$ with maps $m, \Delta, \eta, \vep$
satisfying the above properties.

\med
Now to recover anything resembling a group structure, we must
append such a bialgebra with an antihomomorphism $S : A \lra A$,
satisfying $S(ab) = S(b) S(a)$, for $a,b \in A$~. This map is
defined implicitly via the property~:
\begin{equation} m(S \otimes
\ID) \circ \Delta = m(\ID \otimes S) \circ \Delta = \eta \circ
\vep~~.
\end{equation}
We call $S$ the \emph{antipode map}. A \emph{Hopf algebra} is then
a bialgebra $(A,m, \eta, \Delta, \vep)$ equipped with an antipode
map $S$~.

\med
Commutative and noncommutative Hopf algebras form the backbone of
quantum `groups' and are essential to the generalizations of
symmetry. Indeed, in most respects a quantum `group' is identifiable
with a Hopf algebra. When such algebras are actually
associated with proper groups of matrices there is
considerable scope for their representations on both finite
and infinite dimensional Hilbert spaces.



\begin{thebibliography}{9}

\bibitem{AS}
E. M. Alfsen and F. W. Schultz: \emph{Geometry of State Spaces of Operator Algebras}, Birkh\"auser, Boston--Basel--Berlin (2003).

\bibitem{ICB71}
I. Baianu : Categories, Functors and Automata Theory: A Novel Approach to Quantum Automata through Algebraic--Topological Quantum Computations., \textit{Proceed. 4th Intl. Congress LMPS}, (August-Sept. 1971).

\bibitem{BGB07}
I. C. Baianu, J. F. Glazebrook and R. Brown.: A Non--Abelian, Categorical Ontology of Spacetimes and Quantum Gravity., \emph{Axiomathes} \textbf{17},(3-4): 353-408(2007).

\bibitem{BBGGk8}
I.C.Baianu, R. Brown J.F. Glazebrook, and G. Georgescu, Towards Quantum Non--Abelian Algebraic Topology. \textit{in preparation}, (2008).

\bibitem{BSS}
F.A. Bais, B. J. Schroers and J. K. Slingerland: Broken quantum symmetry and confinement phases in planar physics, \emph{Phys. Rev. Lett.} \textbf{89} No. 18 (1--4): 181--201 (2002).

\bibitem{BJW}
J.W. Barrett.: Geometrical measurements in three-dimensional quantum gravity.
Proceedings of the Tenth Oporto Meeting on Geometry, Topology and Physics (2001).
\textit{Intl. J. Modern Phys.} \textbf{A 18} , October, suppl., 97--113 (2003).

\bibitem{BRM2k3}
M. R. Buneci.: \emph{Groupoid Representations}, (orig. title ``Reprezentari de Grupoizi''),
Ed. Mirton: Timishoara (2003). 

\bibitem{Chaician}
M. Chaician and A. Demichev: \emph{Introduction to Quantum Groups}, World Scientific (1996).

\bibitem{Coleman}
Coleman and De Luccia: Gravitational effects on and of vacuum decay., \emph{Phys. Rev. D} \textbf{21}: 3305 (1980).

\bibitem{CF}
L. Crane and I.B. Frenkel. Four-dimensional topological quantum field theory, Hopf categories, and the canonical bases. Topology and physics. \textit{J. Math. Phys}. \textbf{35} (no. 10): 5136--5154 (1994).

\bibitem{DT96}
W. Drechsler and P. A. Tuckey:  On quantum and parallel transport in a Hilbert bundle over spacetime., \emph{Classical and Quantum Gravity}, \textbf{13}:611-632 (1996).
doi: 10.1088/0264--9381/13/4/004

\bibitem{Drinfeld}
V. G. Drinfel'd: Quantum groups, In \emph{Proc. Intl. Congress of
Mathematicians, Berkeley 1986}, (ed. A. Gleason), Berkeley, 798-820 (1987).

\bibitem{Ellis}
G. J. Ellis: Higher dimensional crossed modules of algebras,
\emph{J. of Pure Appl. Algebra} \textbf{52} (1988), 277-282.

\bibitem{Etingof1}
P.. I. Etingof and A. N. Varchenko, Solutions of the Quantum Dynamical Yang-Baxter Equation and Dynamical Quantum Groups, \emph{Comm.Math.Phys.}, \textbf{196}: 591-640 (1998).

\bibitem{Etingof2}
P. I. Etingof and A. N. Varchenko: Exchange dynamical quantum
groups, \emph{Commun. Math. Phys.} \textbf{205} (1): 19-52 (1999)

\bibitem{Etingof3}
P. I. Etingof and O. Schiffmann: Lectures on the dynamical Yang--Baxter equations, in \emph{Quantum Groups and Lie Theory (Durham, 1999)}, pp. 89-129, Cambridge University Press, Cambridge, 2001.

\bibitem{Fauser2002}
B. Fauser: \emph{A treatise on quantum Clifford Algebras}. Konstanz,
Habilitationsschrift. \\ arXiv.math.QA/0202059 (2002).

\bibitem{Fauser2004}
B. Fauser: Grade Free product Formulae from Grassman--Hopf Gebras.
Ch. 18 in R. Ablamowicz, Ed., \emph{Clifford Algebras: Applications to Mathematics, Physics and Engineering}, Birkh\"{a}user: Boston, Basel and Berlin, (2004).

\bibitem{Fell}
J. M. G. Fell.: The Dual Spaces of  C*--Algebras., \emph{Transactions of the American
Mathematical Society}, \textbf{94}: 365--403 (1960).

\bibitem{FernCastro}
F.M. Fernandez and E. A. Castro.:  \emph{(Lie) Algebraic Methods in Quantum Chemistry and Physics.}, Boca Raton: CRC Press, Inc  (1996).

\bibitem{Feynman}
 R. P. Feynman: Space--Time Approach to Non--Relativistic Quantum Mechanics, {\em Reviews 
of Modern Physics}, 20: 367--387 (1948). [It is also reprinted in (Schwinger 1958).]

\bibitem{frohlich:nonab}
A.~Fr{\"o}hlich: Non--Abelian Homological Algebra. {I}.
{D}erived functors and satellites.\/, \emph{Proc. London Math. Soc.}, \textbf{11}(3): 239--252 (1961).

\bibitem{GR02}
R. Gilmore: \emph{Lie Groups, Lie Algebras and Some of Their Applications.},
Dover Publs., Inc.: Mineola and New York, 2005.

\bibitem{Hahn1}
P. Hahn: Haar measure for measure groupoids., \textit{Trans. Amer. Math. Soc}. \textbf{242}: 1--33(1978).

\bibitem{Hahn2}
P. Hahn: The regular representations of measure groupoids., \textit{Trans. Amer. Math. Soc}. \textbf{242}:34--72(1978).

\bibitem{HeynLifsctz}
R. Heynman and S. Lifschitz. 1958. \emph{Lie Groups and Lie Algebras}., New York and London: Nelson Press.

\bibitem{HLS2k8}
C. Heunen, N. P. Landsman, B. Spitters.: A topos for algebraic quantum theory, (2008)   \\ 
arXiv:0709.4364v2 [quant--ph]

\end{thebibliography}
%%%%%
%%%%%
\end{document}
