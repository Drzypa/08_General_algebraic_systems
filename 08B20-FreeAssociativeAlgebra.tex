\documentclass[12pt]{article}
\usepackage{pmmeta}
\pmcanonicalname{FreeAssociativeAlgebra}
\pmcreated{2013-03-22 16:51:07}
\pmmodified{2013-03-22 16:51:07}
\pmowner{Algeboy}{12884}
\pmmodifier{Algeboy}{12884}
\pmtitle{free associative algebra}
\pmrecord{10}{39098}
\pmprivacy{1}
\pmauthor{Algeboy}{12884}
\pmtype{Definition}
\pmcomment{trigger rebuild}
\pmclassification{msc}{08B20}
\pmrelated{Algebras}
\pmrelated{TensorAlgebra}
\pmdefines{free associative algebra}

\usepackage{latexsym}
\usepackage{amssymb}
\usepackage{amsmath}
\usepackage{amsfonts}
\usepackage{amsthm}

%%\usepackage{xypic}

%-----------------------------------------------------

%       Standard theoremlike environments.

%       Stolen directly from AMSLaTeX sample

%-----------------------------------------------------

%% \theoremstyle{plain} %% This is the default

\newtheorem{thm}{Theorem}

\newtheorem{coro}[thm]{Corollary}

\newtheorem{lem}[thm]{Lemma}

\newtheorem{lemma}[thm]{Lemma}

\newtheorem{prop}[thm]{Proposition}

\newtheorem{conjecture}[thm]{Conjecture}

\newtheorem{conj}[thm]{Conjecture}

\newtheorem{defn}[thm]{Definition}

\newtheorem{remark}[thm]{Remark}

\newtheorem{ex}[thm]{Example}



%\countstyle[equation]{thm}



%--------------------------------------------------

%       Item references.

%--------------------------------------------------


\newcommand{\exref}[1]{Example-\ref{#1}}

\newcommand{\thmref}[1]{Theorem-\ref{#1}}

\newcommand{\defref}[1]{Definition-\ref{#1}}

\newcommand{\eqnref}[1]{(\ref{#1})}

\newcommand{\secref}[1]{Section-\ref{#1}}

\newcommand{\lemref}[1]{Lemma-\ref{#1}}

\newcommand{\propref}[1]{Prop\-o\-si\-tion-\ref{#1}}

\newcommand{\corref}[1]{Cor\-ol\-lary-\ref{#1}}

\newcommand{\figref}[1]{Fig\-ure-\ref{#1}}

\newcommand{\conjref}[1]{Conjecture-\ref{#1}}


% Normal subgroup or equal.

\providecommand{\normaleq}{\unlhd}

% Normal subgroup.

\providecommand{\normal}{\lhd}

\providecommand{\rnormal}{\rhd}
% Divides, does not divide.

\providecommand{\divides}{\mid}

\providecommand{\ndivides}{\nmid}


\providecommand{\union}{\cup}

\providecommand{\bigunion}{\bigcup}

\providecommand{\intersect}{\cap}

\providecommand{\bigintersect}{\bigcap}










\begin{document}
Fix a commutative unital ring $K$ and a set $X$.  Then a $K$-algebra $F$ is
said to be \emph{free on $X$} if there exists an injection $\iota:X\rightarrow F$ such that for all functions $f:X\rightarrow A$ where $A$ is an $K$-algebra
determine a unique algebra homomorphism $\hat{f}:F\rightarrow A$ such that
$\iota\hat{f}=f$.  This is an example of a universal mapping property for
free associative algebras and in categorical settings is often explained
with the following commutative diagram:
\[\xymatrix{
  &  X\ar[ld]_{\iota}\ar[rd]^{f} & \\
F\ar[rr]^{\hat{f}} & & A.
}\]

To prove that free associative algebras exist in the category of all
associative algebras we provide a couple standard constructions.  It
is a standard categorical procedure to conclude any two free objects
on the same set are naturally equivalent and thus each construction
below is equivalent.

\section{Tensor algebra}

Let $X$ be a set and $K$ a commutative unital ring.  Then take $M$ to be
any free $K$-module with basis $X$, and injection $\iota:X\rightarrow M$.  
Then we may form the tensor algebra of $M$, 
   \[T(M)=\bigoplus_{i\in\mathbb{N}} T^i(M),\qquad 
                T^i(M)=M^{\otimes i}= \bigotimes_{j=1}^i M.\]
[Note, $0\in\mathbb{N}$ and the empty tensor we define as $K$.]
Furthermore, define the injection $\iota':X\rightarrow T(M)$ as the map $\iota:X\rightarrow M$ followed by the embedding of $M$ into $T(M)$.

\begin{remark}
To make $M$ concrete use the set of all functions $f:X\rightarrow K$, or 
equivalently, the direct product $\prod_{X} K$.  Then the tensor algebra
of $M$ is the free algebra on $X$.
\end{remark}

\begin{prop}
$(T(M),\iota')$ is a free associative algebra on $X$.
\end{prop}
\begin{proof}
Given any associative $K$-algebra $A$ and function $f:X\rightarrow A$, then 
$A$ is a $K$-module and $M$ is free on $X$ so $f$ extends to a unique 
$K$-linear homomorphism $\hat{f}:M\rightarrow A$.  

Next we define $K$-multilinear maps $f^{(i)}:M^i\rightarrow A$ by 
   \[f^{(i)}(m_1,\dots,m_i)=f(m_1)\cdots f(m_i).\]
Then by the universal mapping property of tensor products (used inductively)
we have a unique $K$-linear map $\hat{f}^{(i)}:T^i(M)\rightarrow A$ for which
   \[\hat{f}^{(i)}(m_1\otimes\cdots\otimes m_i)=f(m_1)\cdots f(m_i).\]
Thus we have a unique algebra homomorphism 
$\hat{f}^{(\infty)}:T(M)\rightarrow A$ such that $\iota\hat{f}=f$.
\end{proof}

This construction provides an obvious grading on the free algebra where
the homogeneous components are 
   \[T^n(M)=M^{\otimes n}=\bigotimes_{j=1}^n M.\]

\section{Non-commutative polynomials}

An alternative construction is to model the methods of constructing free
groups and semi-groups, that is, to use words on the set $X$.  We will
denote the result of this construction by $K\langle X\rangle$ and we will
find many parallels to polynomial algebras with indeterminants in $X$.

Let $FM\langle X\rangle$ be the set of all words on $X$.  This makes
$FM\langle X\rangle$ a free monoid with identity the empty word and 
associative product the juxtaposition of words.  Then define 
$K\langle X\rangle$ as the $K$-semi-group algebra on $FM\langle X\rangle$.
This means $K\langle X\rangle$ is the free $K$-modules oN $FM\langle X\rangle$
and the product is defined as:
   \[\left(\sum_{w\in FM\langle X\rangle} l_w w\right)
     \left(\sum_{v\in FM\langle X\rangle} l_v v\right)
     =\sum_{w,v\in FM\langle X\rangle} l_v l_w wv.\] 

For example, $\mathbb{Q}\langle x,y\rangle$ contains elements of the form
   \[x^2+4yxy,\qquad -7xy+2yx,\qquad 1+x+xy+xyx+x^2y+x^2y^2.\]

This model of a free associative algebra encourages a mapping to polynomial
rings.  Indeed, $K\langle X\rangle\rightarrow K[X]$ is uniquely determined by
the free property applied to the natural inclusion of $X$ into $K[X]$.
What we realize this mapping in a practical fashion we note that this simply
allows all indeterminants to commute.  It follows from this that $K[X]$ is
a free commutative associaitve algebra.

For example, under this map we translate the above elements into:
   \[x^2+4xy^2,\qquad -5xy,\qquad 1+x+xy+2x^2y+x^2y^2.\]


We also note that the grading detected in the tensor algebra construction
persists in the non-commuting polynomial model.  In particular, we say an
element in $K\langle X\rangle$ is homogeneous if it contained in
$FM\langle X\rangle$.  Then the degree of a homogeneous element is the
length of the word.  Then the $K$-linear span of elements of degree $i$
form the $i$-th graded component of $K\langle X\rangle$.


\begin{remark}
We note that the free properties of both of these constructions depend
in turn on the free properties of modules, the universal property of
tensors and free semi-groups.  An inspection of the common construction
of tensors and free modules reveals both of these have universal properties
implied from the universal mapping property of free semi-groups.  Thus
we may assert that free of associative algebras are a direct result of
the existence of free semi-groups.  

For non-associative algebras such as Lie and Jordan algebras, the 
universal properties are more subtle.
\end{remark}


%%%%%
%%%%%
\end{document}
