\documentclass[12pt]{article}
\usepackage{pmmeta}
\pmcanonicalname{DivisionInGroup}
\pmcreated{2013-03-22 15:08:01}
\pmmodified{2013-03-22 15:08:01}
\pmowner{pahio}{2872}
\pmmodifier{pahio}{2872}
\pmtitle{division in group}
\pmrecord{13}{36877}
\pmprivacy{1}
\pmauthor{pahio}{2872}
\pmtype{Theorem}
\pmcomment{trigger rebuild}
\pmclassification{msc}{08A99}
\pmclassification{msc}{20A05}
\pmclassification{msc}{20-00}
\pmrelated{Group}
\pmrelated{Division}
\pmrelated{Groupoid}
\pmrelated{AlternativeDefinitionOfGroup}
\pmdefines{division groupoid}

\endmetadata

% this is the default PlanetMath preamble.  as your knowledge
% of TeX increases, you will probably want to edit this, but
% it should be fine as is for beginners.

% almost certainly you want these
\usepackage{amssymb}
\usepackage{amsmath}
\usepackage{amsfonts}

% used for TeXing text within eps files
%\usepackage{psfrag}
% need this for including graphics (\includegraphics)
%\usepackage{graphicx}
% for neatly defining theorems and propositions
 \usepackage{amsthm}
 \usepackage[T2A]{fontenc}
 \usepackage[russian, english]{babel}
% making logically defined graphics
%%%\usepackage{xypic}

% there are many more packages, add them here as you need them

% define commands here
\theoremstyle{definition}
\newtheorem*{thmplain}{Theorem}
\begin{document}
In any group $(G,\,\cdot)$ one can introduce a {\em division operation} ``:'' by setting
            $$x:y = x \cdot y^{-1}$$
for all elements $x$, $y$ of $G$. \,On the contrary, the group operation and the unary inverse forming operation may be expressed via the division by
\begin{align}
x \cdot y = x:((y:y):y), \quad x^{-1} = (x:x):x.
\end{align}

The division, which of course is not associative, has the properties
\begin{enumerate} 
\item \,$(x:z):(y:z) = x:y,$  
\item \,$x:(y:y) = x,$
\item \,$(x:x):(y:z) = z:y.$
\end{enumerate}

The above result may be conversed:
\begin{thmplain}
\,If the operation ``:'' of the non-empty groupoid $G$ has the properties 1, 2, and 3, then $G$ equipped with the ``multiplication'' and inverse forming by (1) is a group.
\end{thmplain}

{\em Proof.} \,Here we prove only the associativity of ``$\cdot$''. \,First we derive some auxiliary results. \,Using definitions and the properties 1 and 2 we obtain
$$(x:y):y^{-1} = (x:y):((y:y):y) = x:(y:y) = x,$$
$$(x:y^{-1}):y = (x:y^{-1}):((y:y):y^{-1}) = x:(y:y) = x$$
and using the property 3, 
$$(x:y)^{-1} = ((x:y):(x:y)):(x:y) = y:x.$$
Then we get: 
$$(xy)z = (x:y^{-1}):z^{-1} = ((x:y^{-1}):y):(z^{-1}:y) = 
  x:(z^{-1}:y) = x:(y:z^{-1})^{-1} = x(yz)$$


\begin{thebibliography}{7}
\bibitem{AIM} \CYRA. \CYRI. \CYRM\cyra\cyrl\cyrsftsn\cyrc\cyre\cyrv: 
{\em \CYRA\cyrl\cyrg\cyre\cyrb\cyrr\cyra\cyri\cyrch\cyre\cyrs\cyrk\cyri\cyre \,
\cyrs\cyri\cyrs\cyrt\cyre\cyrm\cyrery}. \,\CYRI\cyrz\cyrd\cyra\cyrt\cyre\cyrl\cyrsftsn\cyrs\cyrt\cyrv\cyro \,
``\CYRN\cyra\cyru\cyrk\cyra''. \CYRM\cyro\cyrs\cyrk\cyrv\cyra \,(1970).
\end{thebibliography}
%%%%%
%%%%%
\end{document}
