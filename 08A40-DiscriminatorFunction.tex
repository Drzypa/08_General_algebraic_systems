\documentclass[12pt]{article}
\usepackage{pmmeta}
\pmcanonicalname{DiscriminatorFunction}
\pmcreated{2013-03-22 18:20:58}
\pmmodified{2013-03-22 18:20:58}
\pmowner{CWoo}{3771}
\pmmodifier{CWoo}{3771}
\pmtitle{discriminator function}
\pmrecord{6}{40985}
\pmprivacy{1}
\pmauthor{CWoo}{3771}
\pmtype{Definition}
\pmcomment{trigger rebuild}
\pmclassification{msc}{08A40}
\pmsynonym{switching function}{DiscriminatorFunction}
\pmdefines{ternary discriminator}
\pmdefines{quaternary discriminator}

\usepackage{amssymb,amscd}
\usepackage{amsmath}
\usepackage{amsfonts}
\usepackage{mathrsfs}

% used for TeXing text within eps files
%\usepackage{psfrag}
% need this for including graphics (\includegraphics)
%\usepackage{graphicx}
% for neatly defining theorems and propositions
\usepackage{amsthm}
% making logically defined graphics
%%\usepackage{xypic}
\usepackage{pst-plot}

% define commands here
\newcommand*{\abs}[1]{\left\lvert #1\right\rvert}
\newtheorem{prop}{Proposition}
\newtheorem{thm}{Theorem}
\newtheorem{ex}{Example}
\newcommand{\real}{\mathbb{R}}
\newcommand{\pdiff}[2]{\frac{\partial #1}{\partial #2}}
\newcommand{\mpdiff}[3]{\frac{\partial^#1 #2}{\partial #3^#1}}
\begin{document}
Let $A$ be a non-empty set.  The \emph{ternary discriminator} on $A$ is the ternary operation $t$ on $A$ such that
\begin{displaymath}
t(a,b,c):= \left\{
\begin{array}{ll}
a & \textrm{if } a\ne b,\\
c & \textrm{otherwise.}
\end{array}
\right.
\end{displaymath}
In other words, $t$ is a function that determines whether or not a pair of elements in $A$ are the same, hence the name discriminator.

It is easy to see that, by setting two of the three variables the same, $t$ becomes a constant function: $t(a,b,a)=a$, $t(a,a,b)=b$, and $t(a,b,b)=a$.

More generally, the \emph{quaternary discriminator} or the \emph{switching function} on $A$ is the quaternary operation $q$ on $A$ such that
\begin{displaymath}
q(a,b,c,d):= \left\{
\begin{array}{ll}
d & \textrm{if } a\ne b,\\
c & \textrm{otherwise.}
\end{array}
\right.
\end{displaymath}

However, this generalization is really an equivalent concept in the sense that one can derive one type of discriminator from another: given $q$ above, set $t(a,b,c)=q(a,b,c,a)$.  Conversely, given $t$ above, set $q(a,b,c,d) = t(t(a,b,c),t(a,b,d),d)$.

\textbf{Remark}.  The following ternary functions $t_1,t_2:A^3\to A$ could also serve as discriminator functions:
\begin{displaymath}
t_1(a,b,c):= \left\{
\begin{array}{ll}
b & \textrm{if } a\ne b,\\
c & \textrm{otherwise.}
\end{array}
\right.
\hspace{1cm}
t_2(a,b,c):= \left\{
\begin{array}{ll}
c & \textrm{if } a\ne b,\\
a & \textrm{otherwise.}
\end{array}
\right.
\end{displaymath}
But they are really no different from \emph{the} ternary discriminator $t$:
$$t_1(a,b,c)=t(b,a,c)\quad \mbox{ and }\quad t(a,b,c)=t_1(b,a,c),$$
$$t_2(a,b,c)=t(c,t(a,b,c),a)\quad \mbox{ and }\quad t(a,b,c)=t_2(a,t_2(a,b,c),c).$$

\begin{thebibliography}{9}
\bibitem{gg} G. Gr\"{a}tzer: {\em Universal Algebra}, 2nd Edition, Springer, New York (1978).
\bibitem{bs} S. Burris, H.P. Sankappanavar: {\em A Course in Universal Algebra}, Springer, New York (1981).
\end{thebibliography}
%%%%%
%%%%%
\end{document}
