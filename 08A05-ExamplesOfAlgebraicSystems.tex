\documentclass[12pt]{article}
\usepackage{pmmeta}
\pmcanonicalname{ExamplesOfAlgebraicSystems}
\pmcreated{2013-03-22 18:40:11}
\pmmodified{2013-03-22 18:40:11}
\pmowner{CWoo}{3771}
\pmmodifier{CWoo}{3771}
\pmtitle{examples of algebraic systems}
\pmrecord{16}{41414}
\pmprivacy{1}
\pmauthor{CWoo}{3771}
\pmtype{Example}
\pmcomment{trigger rebuild}
\pmclassification{msc}{08A05}
\pmclassification{msc}{03E99}
\pmclassification{msc}{08A62}

\endmetadata

\usepackage{amssymb,amscd}
\usepackage{amsmath}
\usepackage{amsfonts}
\usepackage{mathrsfs}

% used for TeXing text within eps files
%\usepackage{psfrag}
% need this for including graphics (\includegraphics)
%\usepackage{graphicx}
% for neatly defining theorems and propositions
\usepackage{amsthm}
% making logically defined graphics
%%\usepackage{xypic}
\usepackage{pst-plot}

% define commands here
\newcommand*{\abs}[1]{\left\lvert #1\right\rvert}
\newtheorem{prop}{Proposition}
\newtheorem{thm}{Theorem}
\newtheorem{ex}{Example}
\newcommand{\real}{\mathbb{R}}
\newcommand{\pdiff}[2]{\frac{\partial #1}{\partial #2}}
\newcommand{\mpdiff}[3]{\frac{\partial^#1 #2}{\partial #3^#1}}
\begin{document}
Selected examples of algebraic systems are specified below.

\begin{enumerate}
\item A set is an algebra where $\tau=\varnothing$.
\item A pointed set is an algebra of type $\langle 0\rangle$, where $0$ corresponds to the designated element in the set.
\item An algebra of type $\langle 2\rangle$ is called a groupoid.  Another algebra of this type is a semigroup.
\item A monoid is an algebra of type $\langle 2,0\rangle$.  However, not every algebra of type $\langle 2,0\rangle$ is a monoid.
\item A group is an algebraic system of type $\langle 2,1,0\rangle$, where $ 2$ corresponds to the arity of the multiplication, $1$ the multiplicative inverse, and $0$ the multiplicative identity.
\item A ring is an algebraic system of type $\langle 2,2,1,0,0\rangle$, where the two $2$'s represent the arities of  addition and multiplication, $1$ the additive inverse, and $0$'s the additive and multiplicative identities. 
\item A lattice is an algebraic system of type $\langle 2,2\rangle$.  The two binary operations are meet and join.
\item A bounded lattice is an algebraic system of type $\langle 2,2,0,0\rangle$.  Besides the meet and join operations, it has two constants, its top $1$ and bottom $0$.
\item A uniquely complemented lattice is an algebraic system of the type $\langle 2,2,1,0,0\rangle$.  In addition to having the operations of a bounded lattice, there is a unary operator taking each element to its unique complement.  Note that it has the same type as the type of a group.
\item A quandle is an algebraic system of type $\langle 2,2,\rangle$.  It has the same type as a lattice.
\item A quasigroup may be thought of as a algebraic system of type $\langle 2\rangle$, that of a groupoid, or $\langle 2,2,2\rangle$, depending on the definition used.  A loop, as a quasigroup with an identity, is an algebraic system of type $\langle 0,q\rangle$, where $q$ is the type of a quasigroup.
\item An \PMlinkname{$n$-group}{PolyadicSemigroup} is an algebraic system of type $\langle n\rangle$.
\item A left module over a ring $R$ is an algebraic system.  Its type is $\langle 2, 1, (1)_{r\in R}, 0\rangle$, where $2$ is the arity of addition, the first $1$ the additive inverse, and the rest of the $1$'s represent the arity of left scalar multiplication by $r$, for each $r\in R$, and finally $0$ the (arity) of additive identity.
\item The set $\overline{V}$ of all well-formed formulas over a set $V$ of propositional variables can be thought of as an algebraic system, as each of the logical connectives as an operation on $\overline{V}$ may be associated with a finitary operation on $\overline{V}$.  In classical propositional logic, the algebraic system may be of type $\langle 1,2\rangle$, if we consider $\neg$ and $\vee$ as the only logical connectives; or it may be of type $\langle 1,2,2,2,2\rangle$, if the full set $\lbrace \neg, \vee, \wedge, \to, \leftrightarrow \rbrace$ is used.
\end{enumerate}

Below are some non-examples of algebraic systems:

\begin{enumerate}
\item A complete lattice is not, in general, an algebraic system because the arbitrary meet and join operations are not finitary.
\item A field is not an algebraic system, since, in addition to the five operations of a ring, there is the multiplicative inverse operation, which is not defined for $0$.
\item A small category may be defined as a set with one partial binary operation on it.  Unless the category has only one object (so that the operation is everywhere defined), it is in general not an algebraic system.
\end{enumerate}

\begin{thebibliography}{7}
\bibitem{gg} G. Gr\"{a}tzer: {\em Universal Algebra}, 2nd Edition, Springer, New York (1978).
\bibitem{pj} P. Jipsen: {\em \PMlinkexternal{Mathematical Structures: Homepage}{http://math.chapman.edu/cgi-bin/structures?HomePage}}
\end{thebibliography}
%%%%%
%%%%%
\end{document}
