\documentclass[12pt]{article}
\usepackage{pmmeta}
\pmcanonicalname{RestrictedDirectProductOfAlgebraicSystems}
\pmcreated{2013-03-22 17:05:57}
\pmmodified{2013-03-22 17:05:57}
\pmowner{CWoo}{3771}
\pmmodifier{CWoo}{3771}
\pmtitle{restricted direct product of algebraic systems}
\pmrecord{7}{39395}
\pmprivacy{1}
\pmauthor{CWoo}{3771}
\pmtype{Definition}
\pmcomment{trigger rebuild}
\pmclassification{msc}{08B25}
\pmdefines{restricted direct product}
\pmdefines{weak direct product}

\usepackage{amssymb,amscd}
\usepackage{amsmath}
\usepackage{amsfonts}
\usepackage{mathrsfs}

% used for TeXing text within eps files
%\usepackage{psfrag}
% need this for including graphics (\includegraphics)
%\usepackage{graphicx}
% for neatly defining theorems and propositions
\usepackage{amsthm}
% making logically defined graphics
%%\usepackage{xypic}
\usepackage{pst-plot}
\usepackage{psfrag}

% define commands here
\newtheorem{prop}{Proposition}
\newtheorem{thm}{Theorem}
\newtheorem{ex}{Example}
\newcommand{\real}{\mathbb{R}}
\newcommand{\pdiff}[2]{\frac{\partial #1}{\partial #2}}
\newcommand{\mpdiff}[3]{\frac{\partial^#1 #2}{\partial #3^#1}}
\begin{document}
Let $\lbrace A_i\mid i\in I\rbrace$ be a family of algebraic systems indexed by a set $I$.  Let $J$ be a Boolean ideal in $P(I)$, the Boolean algebra over the power set of $I$.  A subset $B$ of the direct product $\prod \lbrace A_i\mid i\in I\rbrace$ is called a \emph{restricted direct product} of $A_i$ if
\begin{enumerate}
\item $B$ is a subalgebra of $\prod \lbrace A_i\mid i\in I\rbrace$, and
\item given any $(a_i)\in B$, we have that $(b_i)\in B$ iff $\lbrace i\in I\mid a_i\ne b_i\rbrace \in J$.
\end{enumerate}
If it is necessary to distinguish the different restricted direct products of $A_i$, we often specify the ``restriction'', hence we say that $B$ is a $J$-restricted direct product of $A_i$, or that $B$ is restricted to $J$.

Here are some special restricted direct products:
\begin{itemize}
\item
If $J=P(I)$ above, then $B$ is the direct product $\prod A_i$, for if $(b_i)\in \prod A_i$, then clearly $\lbrace i\in I\mid a_i\ne b_i\rbrace\in P(I)$, where $(a_i)\in B$ ($B$ is non-empty since it is a subalgebra).  Therefore $(b_i)\in B$.

This justifies calling the direct product the ``unrestricted direct product'' by some people.
\item
If $J$ is the ideal consisting of all finite subsets of $I$, then $B$ is called the \emph{weak direct product} of $A_i$.
\item
If $J$ is the singleton $\lbrace \varnothing\rbrace$, then $B$ is also a singleton: pick $a,b\in B$, then $\lbrace i\mid a_i\ne b_i\rbrace = \varnothing$, which is equivalent to saying that $(a_i)=(b_i)$.
\end{itemize}

\textbf{Remark}.  While the direct product of $A_i$ always exists, restricted direct products may not.  For example, in the last case above, A $\varnothing$-restricted direct product exists only when there is an element $a\in \prod A_i$ that is fixed by all operations on it: that is, if $f$ is an $n$-ary operation on $\prod A_i$, then $f(a,\ldots,a)=a$.  In this case, $\lbrace a\rbrace$ is a $\varnothing$-restricted direct product of $\prod A_i$.

\begin{thebibliography}{6}
\bibitem{gg} G. Gr\"{a}tzer: {\em Universal Algebra}, 2nd Edition, Springer, New York (1978).
\end{thebibliography}
%%%%%
%%%%%
\end{document}
