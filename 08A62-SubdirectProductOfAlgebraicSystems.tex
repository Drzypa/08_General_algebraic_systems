\documentclass[12pt]{article}
\usepackage{pmmeta}
\pmcanonicalname{SubdirectProductOfAlgebraicSystems}
\pmcreated{2013-03-22 16:44:51}
\pmmodified{2013-03-22 16:44:51}
\pmowner{CWoo}{3771}
\pmmodifier{CWoo}{3771}
\pmtitle{subdirect product of algebraic systems}
\pmrecord{10}{38972}
\pmprivacy{1}
\pmauthor{CWoo}{3771}
\pmtype{Definition}
\pmcomment{trigger rebuild}
\pmclassification{msc}{08A62}
\pmclassification{msc}{08A05}
\pmclassification{msc}{08B26}
\pmdefines{subdirect product}
\pmdefines{subdirect power}
\pmdefines{subdirectly irreducible}
\pmdefines{trivial subdirect product}

\usepackage{amssymb,amscd}
\usepackage{amsmath}
\usepackage{amsfonts}

% used for TeXing text within eps files
%\usepackage{psfrag}
% need this for including graphics (\includegraphics)
%\usepackage{graphicx}
% for neatly defining theorems and propositions
\usepackage{amsthm}
% making logically defined graphics
%%\usepackage{xypic}
\usepackage{pst-plot}
\usepackage{psfrag}

% define commands here
\newtheorem{prop}{Proposition}
\newtheorem{thm}{Theorem}
\newtheorem{ex}{Example}
\newcommand{\real}{\mathbb{R}}
\begin{document}
In this entry, all algebraic systems are of the same type.  For each algebraic system, we drop the associated operator set for simplicity.

Let $A_i$ be algebraic systems indexed by $i\in I$.  $B$ is called a \emph{subdirect product} of $A_i$ if
\begin{enumerate}
\item $B$ is a subalgebra of the direct product of $A_i$.
\item for each $i\in I$, $\pi_i(B)=A_i$.
\end{enumerate}
In the second condition, $\pi_i$ denotes the projection homomorphism $\prod A_i \to A_i$.  By restriction, we may consider $\pi_i$ as homomorphisms $B\to A_i$.  When $B$ is isomorphic to $\prod A_i$, then $B$ is a \emph{trivial subdirect product} of $A_i$.

This generalizes the notion of a direct product, since in many instances, an algebraic system can not be decomposed into a direct product of algebras.

When all $A_i=C$ for some algebraic system $C$ of the same type, then $B$ is called a \emph{subdirect power} of $C$.

\textbf{Remarks}.  
\begin{enumerate}
\item
A very simple example of a subdirect product is the following: let $A_1=A_2=\lbrace 1,2,3\rbrace$.  Then the subset $B=\lbrace (x,y)\in A_1\times A_2 \mid x\le y \rbrace$ is a subdirect product of the sets $A_1$ and $A_2$ (considered as algebraic systems with no operators).
\item 
Let $B$ is a subdirect product of $A_i$, and $p_i:=(\pi_i)_B$, the restriction of $\pi_i$ to $B$.  Then $B/\ker(p_i)\cong A_i$.  In addition, $$\bigcap \lbrace \ker(p_i)\mid i\in I\rbrace=\Delta,$$ where $\Delta$ is the diagonal relation.  To see the last equality, suppose $a,b\in B$ with $a\equiv b \pmod {p_i}$.  Then $a(i)=\pi_i(a)=p_i(a)=p_i(b)=\pi_i(b)=b(i)$.  Since this is true for every $i\in I$, $a=b$.
\item 
Conversely, if $A$ is an algebraic system and $\lbrace \mathfrak{C}_i\mid i\in I\rbrace$ is a set of congruences on $A$ such that $$\bigcap \lbrace \mathfrak{C}_i\mid i\in I\rbrace=\Delta.$$ 
Then $A$ is isomorphic to a subdirect product of $A/\mathfrak{C}_i$.
\item
An algebraic system is said to be \emph{subdirectly irreducible} if, whenever $\mathfrak{C}_i$ are congruences on $A$ and $\bigcap \lbrace \mathfrak{C}_i\mid i\in I\rbrace=\Delta$, then one of $\mathfrak{C}_i=\Delta$.
\item \textbf{Birkhoff's Theorem on the Decomposition of an Algebraic System}.  Every algebraic system is isomorphic to a subdirect product of subdirectly irreducible algebraic systems.  This works only when the algebraic system is finitary.
\end{enumerate}
%%%%%
%%%%%
\end{document}
