\documentclass[12pt]{article}
\usepackage{pmmeta}
\pmcanonicalname{FilteredAlgebra}
\pmcreated{2013-03-22 13:23:55}
\pmmodified{2013-03-22 13:23:55}
\pmowner{Dr_Absentius}{537}
\pmmodifier{Dr_Absentius}{537}
\pmtitle{filtered algebra}
\pmrecord{11}{33938}
\pmprivacy{1}
\pmauthor{Dr_Absentius}{537}
\pmtype{Definition}
\pmcomment{trigger rebuild}
\pmclassification{msc}{08A99}

%\documentclass{amsart}
\usepackage{amsmath}
%\usepackage[all,poly,knot,dvips]{xy}
%\usepackage{pstricks,pst-poly,pst-node,pstcol}


\usepackage{amssymb,latexsym}

\usepackage{amsthm,latexsym}
\usepackage{eucal,latexsym}

% THEOREM Environments --------------------------------------------------

\newtheorem{thm}{Theorem}
 \newtheorem*{mainthm}{Main~Theorem}
 \newtheorem{cor}[thm]{Corollary}
 \newtheorem{lem}[thm]{Lemma}
 \newtheorem{prop}[thm]{Proposition}
 \newtheorem{claim}[thm]{Claim}
 \theoremstyle{definition}
 \newtheorem{defn}[thm]{Definition}
 \theoremstyle{remark}
 \newtheorem{rem}[thm]{Remark}
 \numberwithin{equation}{subsection}


%---------------------  Greek letters, etc ------------------------- 

\newcommand{\CA}{\mathcal{A}}
\newcommand{\CC}{\mathcal{C}}
\newcommand{\CM}{\mathcal{M}}
\newcommand{\CP}{\mathcal{P}}
\newcommand{\CS}{\mathcal{S}}
\newcommand{\BC}{\mathbb{C}}
\newcommand{\BN}{\mathbb{N}}
\newcommand{\BR}{\mathbb{R}}
\newcommand{\BZ}{\mathbb{Z}}
\newcommand{\FF}{\mathfrak{F}}
\newcommand{\FL}{\mathfrak{L}}
\newcommand{\FM}{\mathfrak{M}}
\newcommand{\Ga}{\alpha}
\newcommand{\Gb}{\beta}
\newcommand{\Gg}{\gamma}
\newcommand{\GG}{\Gamma}
\newcommand{\Gd}{\delta}
\newcommand{\GD}{\Delta}
\newcommand{\Ge}{\varepsilon}
\newcommand{\Gz}{\zeta}
\newcommand{\Gh}{\eta}
\newcommand{\Gq}{\theta}
\newcommand{\GQ}{\Theta}
\newcommand{\Gi}{\iota}
\newcommand{\Gk}{\kappa}
\newcommand{\Gl}{\lambda}
\newcommand{\GL}{\Lamda}
\newcommand{\Gm}{\mu}
\newcommand{\Gn}{\nu}
\newcommand{\Gx}{\xi}
\newcommand{\GX}{\Xi}
\newcommand{\Gp}{\pi}
\newcommand{\GP}{\Pi}
\newcommand{\Gr}{\rho}
\newcommand{\Gs}{\sigma}
\newcommand{\GS}{\Sigma}
\newcommand{\Gt}{\tau}
\newcommand{\Gu}{\upsilon}
\newcommand{\GU}{\Upsilon}
\newcommand{\Gf}{\varphi}
\newcommand{\GF}{\Phi}
\newcommand{\Gc}{\chi}
\newcommand{\Gy}{\psi}
\newcommand{\GY}{\Psi}
\newcommand{\Gw}{\omega}
\newcommand{\GW}{\Omega}
\newcommand{\Gee}{\epsilon}
\newcommand{\Gpp}{\varpi}
\newcommand{\Grr}{\varrho}
\newcommand{\Gff}{\phi}
\newcommand{\Gss}{\varsigma}

\def\co{\colon\thinspace}
\begin{document}
\begin{defn}
A filtered algebra over the field $k$ is an algebra $(A,\cdot)$ over $k$
which is endowed with a filtration $\mathcal{F}=\{F_i\}_{i\in \mathbb{N}}$
by subspaces,  compatible with the multiplication in
the following sense
$$\forall m,n \in \mathbb{N},\qquad F_m\cdot F_n\subset F_{n+m}.$$
\end{defn}

A special case of filtered algebra is a graded algebra. In general there is
the following construction that produces a graded algebra out of a filtered
algebra.

\begin{defn}
Let $(A,\cdot,\mathcal{F})$ be a filtered algebra then the \emph{associated \PMlinkid{graded algebra}{3071}} $ \mathcal{G}(A)$ is defined as follows:
\begin{itemize}
\item As a vector space  
$$ \mathcal{G}(A)=\bigoplus_{n\in \mathbb{N}}G_n\,, $$
where,
$$G_0=F_0,\quad \text{and } \forall n>0, \quad G_n=F_n/F_{n-1}\,,$$
\item the multiplication is defined by
$$(x+F_{n})(y+F_{m})=x\cdot y+F_{n+m}$$   
\end{itemize}
\end{defn}

\begin{thm}
The multiplication is well defined and endows $\mathcal{G}(A)$ with the 
\PMlinkescapetext{structure} of a graded algebra, with gradation 
$\{G_n\}_{n \in \mathbb{N}}$.
Furthermore if  $A$ is associative then so is $\mathcal{G}(A)$.
\end{thm}

An example of a filtered algebra is the Clifford algebra $\mathrm{Cliff}(V,q)$
of a vector space $V$ endowed with a quadratic form $q$. The associated
graded algebra is $\bigwedge V$, the exterior algebra of $V$.


As algebras
$A$ and $\mathcal{G}(A)$ are distinct (with the exception of the trivial
case that $A$ is graded) but as vector spaces they are isomorphic.  

\begin{thm}
 The underlying vector spaces of $A$ and $\mathcal{G}(A)$ are isomorphic. 
\end{thm}
%%%%%
%%%%%
\end{document}
