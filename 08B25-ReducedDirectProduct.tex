\documentclass[12pt]{article}
\usepackage{pmmeta}
\pmcanonicalname{ReducedDirectProduct}
\pmcreated{2013-03-22 17:10:11}
\pmmodified{2013-03-22 17:10:11}
\pmowner{CWoo}{3771}
\pmmodifier{CWoo}{3771}
\pmtitle{reduced direct product}
\pmrecord{10}{39482}
\pmprivacy{1}
\pmauthor{CWoo}{3771}
\pmtype{Definition}
\pmcomment{trigger rebuild}
\pmclassification{msc}{08B25}
\pmsynonym{ultraproduct}{ReducedDirectProduct}
\pmdefines{prime product}

\endmetadata

\usepackage{amssymb,amscd}
\usepackage{amsmath}
\usepackage{amsfonts}
\usepackage{mathrsfs}

% used for TeXing text within eps files
%\usepackage{psfrag}
% need this for including graphics (\includegraphics)
%\usepackage{graphicx}
% for neatly defining theorems and propositions
\usepackage{amsthm}
% making logically defined graphics
%%\usepackage{xypic}
\usepackage{pst-plot}
\usepackage{psfrag}

% define commands here
\newtheorem{prop}{Proposition}
\newtheorem{thm}{Theorem}
\newtheorem{lem}{Lemma}
\newtheorem{ex}{Example}
\newcommand{\real}{\mathbb{R}}
\newcommand{\pdiff}[2]{\frac{\partial #1}{\partial #2}}
\newcommand{\mpdiff}[3]{\frac{\partial^#1 #2}{\partial #3^#1}}
\begin{document}
Let $\lbrace A_i\mid i\in I\rbrace$ be a set of algebraic systems of the same type, indexed by $I$.  Let $A$ be the direct product of the $A_i$'s.  For any $a,b\in A$, set $$\operatorname{supp}(a,b):=\lbrace k\in I\mid a(k)\ne b(k)\rbrace.$$  Consider a Boolean ideal $L$ of the Boolean algebra $P(I)$ of $I$.  Define a binary relation $\Theta_L$ on $A$ as follows:
$$(a,b)\in \Theta_L\quad \mbox{ iff }\quad \operatorname{supp}(a,b)\in L.$$

\begin{lem} $\Theta_L$ defined above is a congruence relation on $A$. \end{lem}
\begin{proof}  Since $L$ is an ideal $\varnothing\in L$.  Therefore, $(a,a)\in \Theta_L$, since $\lbrace k\in I\mid a(k)\ne a(k)\rbrace=\varnothing$.  Clearly, $\Theta_L$ is symmetric.  For transitivity, suppose  $(a,b,(b,c)\in \Theta_L$.  If $a(k)\ne c(k)$ for some $k\in I$, then either $a(k)\ne b(k)$ or $b(k)\ne c(k)$ (a contrapositive argument).  So $$\operatorname{supp}(a,c) \subseteq \operatorname{supp}(a,b) \cup \operatorname{supp}(b,c).$$  Since $L$ is an ideal, $\operatorname{supp}(a,c)\in L$, so $(a,c)\in \Theta_L$, and $\Theta_L$ is an equivalence relation on $A$.

Next, let $\omega$ be an $n$-ary operator on $A$ and $a_j\equiv b_j\pmod {\Theta_L}$, where $j=1,\ldots,n$.  We want to show that $\omega(a_1,\ldots,a_n)\equiv \omega(b_1,\ldots,b_n)\pmod {\Theta_L}$.  Let $\omega_i$ be the associated $n$-ary operators on $A_i$.  If $\omega(a_1,\ldots,a_n)(k)\ne \omega(b_1,\ldots,b_n)(k)$, then $\omega_k(a_1(k),\ldots,a_n(k))\ne \omega_k(b_1(k),\ldots,b_n(k))$, which implies that $a_j(k)\ne b_j(k)$ for some $j=1,\ldots,n$.  This implies that $$\operatorname{supp}(\omega(a_1,\ldots,a_n),\omega(b_1,\ldots,b_n))\subseteq \bigcup_{j=1}^n \operatorname{supp}(a_j,b_j).$$
Since $L$ is an ideal, and each $\operatorname{supp}(a_j,b_j)\in L$, we have that $\operatorname{supp}(\omega(a_1,\ldots,a_n),\omega(b_1,\ldots,b_n))\in L$ as well, this means that $\omega(a_1,\ldots,a_n)\equiv \omega(b_1,\ldots,b_n)\pmod {\Theta_L}$.
\end{proof}

\textbf{Definition}.  Let $A=\prod \lbrace A_i\mid i\in I\rbrace$, $L$ be a Boolean ideal of $P(I)$ and $\Theta_L$ be defined as above.  The quotient algebra $A/\Theta_L$ is called the $L$-\emph{reduced direct product} of $A_i$.  The $L$-reduced direct product of $A_i$ is denoted by $\prod_L \lbrace A_i\mid i\in I\rbrace$.  Given any element $a\in A$, its image in the reduced direct product $\prod_L \lbrace A_i\mid i\in I\rbrace$ is given by $[a]\Theta_L$, or $[a]$ for short.

\textbf{Example}.  Let $A=A_1\times \cdots \times A_n$, and let $L$ be the principal ideal generated by $1$.  Then $L=\lbrace \varnothing, \lbrace 1\rbrace\rbrace$.  The congruence $\Theta_L$ is given by $(a_1,\ldots,a_n)\equiv (b_1,\ldots, b_n)\pmod {\Theta_L}$ iff $\lbrace i\mid a_i\ne b_i\rbrace =\varnothing$ or $\lbrace 1\rbrace$.  This implies that $a_i=b_i$ for all $i=2,\ldots,n$.  In other words, $\Theta_L$ is isomorphic to the direct product of $A_2\times\cdots\times A_n$.  Therefore, the $L$-reduced direct product of $A_i$ is isomorphic to $A_1$.  

The example above can be generalized:  if $J\subseteq I$, then $$\prod {}_{P(J)} \lbrace A_i\mid i\in I\rbrace \cong \prod \lbrace A_i\mid i\in I-J\rbrace.$$  For $a\in A=\prod \lbrace A_i\mid i\in I\rbrace$, write $a=(a_i)_{i\in I}$.  It is not hard to see that the map $f:\prod_{P(J)} \lbrace A_i\mid i\in I\rbrace \to \prod \lbrace A_i\mid i\in I-J\rbrace$ given by $f([a])=(a_i)_{i\in I-J}$ is the required isomorphism.

\textbf{Remark}.  The definition of a reduced direct product in terms of a Boolean ideal can be equivalently stated in terms of a Boolean filter $F$.  All there is to do is to replace $\operatorname{supp}(a,b)$ by its complement: $\operatorname{supp}(a,b)^c:=\lbrace k\in I\mid a(k)=b(k)\rbrace$.  The congruence relation is now $\Theta_{F'}$, where $F'=\lbrace I-J \mid J\in F\rbrace$ is the ideal complement of $F$.  When $F$ is prime, the $F'$-reduced direct product is called a \emph{prime product}, or an \emph{ultraproduct}, since any prime filter is also called an ultrafilter.  Ultraproducts can be more generally defined over arbitrary structures.

\begin{thebibliography}{7}
\bibitem{gg} G. Gr\"{a}tzer: {\em Universal Algebra}, 2nd Edition, Springer, New York (1978).
\end{thebibliography}
%%%%%
%%%%%
\end{document}
