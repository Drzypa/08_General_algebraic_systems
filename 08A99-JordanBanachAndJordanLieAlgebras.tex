\documentclass[12pt]{article}
\usepackage{pmmeta}
\pmcanonicalname{JordanBanachAndJordanLieAlgebras}
\pmcreated{2013-03-22 18:14:05}
\pmmodified{2013-03-22 18:14:05}
\pmowner{bci1}{20947}
\pmmodifier{bci1}{20947}
\pmtitle{Jordan-Banach and Jordan-Lie algebras}
\pmrecord{32}{40823}
\pmprivacy{1}
\pmauthor{bci1}{20947}
\pmtype{Topic}
\pmcomment{trigger rebuild}
\pmclassification{msc}{08A99}
\pmclassification{msc}{08A05}
\pmclassification{msc}{08A70}
\pmsynonym{quantum operator algebras}{JordanBanachAndJordanLieAlgebras}
%\pmkeywords{Jordan-Banach}
%\pmkeywords{Jordan-Lie algebras}
%\pmkeywords{Jordan-Banach-Lie algebra}
%\pmkeywords{Jordan algebras in quantum operator theory and quantum algebraic topology}
%\pmkeywords{Jacobian matrices}
\pmrelated{Algebras2}
\pmrelated{CAlgebra3}
\pmrelated{AlgebraicCategoryOfLMnLogicAlgebras}
\pmrelated{NonAbelianStructures}
\pmrelated{AbelianCategory}
\pmrelated{AxiomsForAnAbelianCategory}
\pmrelated{GeneralizedVanKampenTheoremsHigherDimensional}
\pmrelated{AxiomaticTheoryOfSupercategories}
\pmrelated{AlgebraicCategoryOfLMnLogicAlgebras}
\pmrelated{Categorical}
\pmdefines{Jordan algebra}
\pmdefines{Jordan-Banach algebra}
\pmdefines{Jordan-Lie algebra}

% this is the default PlanetMath preamble.  as your knowledge
% of TeX increases, you will probably want to edit this, but
% it should be fine as is for beginners.

% almost certainly you want these
\usepackage{amssymb}
\usepackage{amsmath}
\usepackage{amsfonts}

% used for TeXing text within eps files
%\usepackage{psfrag}
% need this for including graphics (\includegraphics)
%\usepackage{graphicx}
% for neatly defining theorems and propositions
%\usepackage{amsthm}
% making logically defined graphics
%%%\usepackage{xypic}

% there are many more packages, add them here as you need them

% define commands here
% this is the default PlanetMath preamble.  as your knowledge
% of TeX increases, you will probably want to edit this, but
% it should be fine as is for beginners.

% almost certainly you want these
\usepackage{amssymb}
\usepackage{amsmath}
\usepackage{amsfonts}

% used for TeXing text within eps files
%\usepackage{psfrag}
% need this for including graphics (\includegraphics)
%\usepackage{graphicx}
% for neatly defining theorems and propositions
%\usepackage{amsthm}
% making logically defined graphics
%%%\usepackage{xypic}

% there are many more packages, add them here as you need them

% define commands here
\usepackage{amssymb,amscd}
\usepackage{amsmath}
\usepackage{amsfonts}
\usepackage{mathrsfs}

% used for TeXing text within eps files
%\usepackage{psfrag}
% need this for including graphics (\includegraphics)
%\usepackage{graphicx}
% for neatly defining theorems and propositions
\usepackage{amsthm}
% making logically defined graphics
%%\usepackage{xypic}
\usepackage{pst-plot}

% define commands here
\newcommand*{\abs}[1]{\left\lvert #1\right\rvert}
\newtheorem{prop}{Proposition}
\newtheorem{thm}{Theorem}
\newtheorem{ex}{Example}
\newcommand{\real}{\mathbb{R}}
\newcommand{\pdiff}[2]{\frac{\partial #1}{\partial #2}}
\newcommand{\mpdiff}[3]{\frac{\partial^#1 #2}{\partial #3^#1}}
\usepackage{amsmath, amssymb, amsfonts, amsthm, amscd, latexsym}
%%\usepackage{xypic}
\usepackage[mathscr]{eucal}

\setlength{\textwidth}{6.5in}
%\setlength{\textwidth}{16cm}
\setlength{\textheight}{9.0in}
%\setlength{\textheight}{24cm}

\hoffset=-.75in     %%ps format
%\hoffset=-1.0in     %%hp format
\voffset=-.4in

\theoremstyle{plain}
\newtheorem{lemma}{Lemma}[section]
\newtheorem{proposition}{Proposition}[section]
\newtheorem{theorem}{Theorem}[section]
\newtheorem{corollary}{Corollary}[section]

\theoremstyle{definition}
\newtheorem{definition}{Definition}[section]
\newtheorem{example}{Example}[section]
%\theoremstyle{remark}
\newtheorem{remark}{Remark}[section]
\newtheorem*{notation}{Notation}
\newtheorem*{claim}{Claim}

\renewcommand{\thefootnote}{\ensuremath{\fnsymbol{footnote%%@
}}}
\numberwithin{equation}{section}

\newcommand{\Ad}{{\rm Ad}}
\newcommand{\Aut}{{\rm Aut}}
\newcommand{\Cl}{{\rm Cl}}
\newcommand{\Co}{{\rm Co}}
\newcommand{\DES}{{\rm DES}}
\newcommand{\Diff}{{\rm Diff}}
\newcommand{\Dom}{{\rm Dom}}
\newcommand{\Hol}{{\rm Hol}}
\newcommand{\Mon}{{\rm Mon}}
\newcommand{\Hom}{{\rm Hom}}
\newcommand{\Ker}{{\rm Ker}}
\newcommand{\Ind}{{\rm Ind}}
\newcommand{\IM}{{\rm Im}}
\newcommand{\Is}{{\rm Is}}
\newcommand{\ID}{{\rm id}}
\newcommand{\GL}{{\rm GL}}
\newcommand{\Iso}{{\rm Iso}}
\newcommand{\Sem}{{\rm Sem}}
\newcommand{\St}{{\rm St}}
\newcommand{\Sym}{{\rm Sym}}
\newcommand{\SU}{{\rm SU}}
\newcommand{\Tor}{{\rm Tor}}
\newcommand{\U}{{\rm U}}

\newcommand{\A}{\mathcal A}
\newcommand{\Ce}{\mathcal C}
\newcommand{\D}{\mathcal D}
\newcommand{\E}{\mathcal E}
\newcommand{\F}{\mathcal F}
\newcommand{\G}{\mathcal G}
\newcommand{\Q}{\mathcal Q}
\newcommand{\R}{\mathcal R}
\newcommand{\cS}{\mathcal S}
\newcommand{\cU}{\mathcal U}
\newcommand{\W}{\mathcal W}

\newcommand{\bA}{\mathbb{A}}
\newcommand{\bB}{\mathbb{B}}
\newcommand{\bC}{\mathbb{C}}
\newcommand{\bD}{\mathbb{D}}
\newcommand{\bE}{\mathbb{E}}
\newcommand{\bF}{\mathbb{F}}
\newcommand{\bG}{\mathbb{G}}
\newcommand{\bK}{\mathbb{K}}
\newcommand{\bM}{\mathbb{M}}
\newcommand{\bN}{\mathbb{N}}
\newcommand{\bO}{\mathbb{O}}
\newcommand{\bP}{\mathbb{P}}
\newcommand{\bR}{\mathbb{R}}
\newcommand{\bV}{\mathbb{V}}
\newcommand{\bZ}{\mathbb{Z}}

\newcommand{\bfE}{\mathbf{E}}
\newcommand{\bfX}{\mathbf{X}}
\newcommand{\bfY}{\mathbf{Y}}
\newcommand{\bfZ}{\mathbf{Z}}

\renewcommand{\O}{\Omega}
\renewcommand{\o}{\omega}
\newcommand{\vp}{\varphi}
\newcommand{\vep}{\varepsilon}

\newcommand{\diag}{{\rm diag}}
\newcommand{\grp}{{\mathbb G}}
\newcommand{\dgrp}{{\mathbb D}}
\newcommand{\desp}{{\mathbb D^{\rm{es}}}}
\newcommand{\Geod}{{\rm Geod}}
\newcommand{\geod}{{\rm geod}}
\newcommand{\hgr}{{\mathbb H}}
\newcommand{\mgr}{{\mathbb M}}
\newcommand{\ob}{{\rm Ob}}
\newcommand{\obg}{{\rm Ob(\mathbb G)}}
\newcommand{\obgp}{{\rm Ob(\mathbb G')}}
\newcommand{\obh}{{\rm Ob(\mathbb H)}}
\newcommand{\Osmooth}{{\Omega^{\infty}(X,*)}}
\newcommand{\ghomotop}{{\rho_2^{\square}}}
\newcommand{\gcalp}{{\mathbb G(\mathcal P)}}

\newcommand{\rf}{{R_{\mathcal F}}}
\newcommand{\glob}{{\rm glob}}
\newcommand{\loc}{{\rm loc}}
\newcommand{\TOP}{{\rm TOP}}

\newcommand{\wti}{\widetilde}
\newcommand{\what}{\widehat}

\renewcommand{\a}{\alpha}
\newcommand{\be}{\beta}
\newcommand{\ga}{\gamma}
\newcommand{\Ga}{\Gamma}
\newcommand{\de}{\delta}
\newcommand{\del}{\partial}
\newcommand{\ka}{\kappa}
\newcommand{\si}{\sigma}
\newcommand{\ta}{\tau}

\newcommand{\med}{\medbreak}
\newcommand{\medn}{\medbreak \noindent}
\newcommand{\bign}{\bigbreak \noindent}
\newcommand{\lra}{{\longrightarrow}}
\newcommand{\ra}{{\rightarrow}}
\newcommand{\rat}{{\rightarrowtail}}
\newcommand{\oset}[1]{\overset {#1}{\ra}}
\newcommand{\osetl}[1]{\overset {#1}{\lra}}
\newcommand{\hr}{{\hookrightarrow}}
\begin{document}
\subsubsection{Definitions of Jordan-Banach, Jordan-Lie, and Jordan-Banach-Lie algebras}

Firstly, a specific \emph{algebra} consists of a vector space $E$ over a ground field (typically $\bR$ or $\bC$) 
equipped with a bilinear and distributive multiplication $\circ$~. Note that $E$ is not 
necessarily commutative or associative.
\med
A \emph{Jordan algebra} (over $\bR$), is an algebra over $\bR$ for which:
\med 

$ \begin{aligned} S \circ T &= T \circ S~, \\ S \circ (T \circ S^2) &= (S \circ T) \circ S^2 
\end{aligned}$, 


\med
for all elements $S, T$ of the algebra.
\med
It is worthwhile noting now that in the algebraic theory of Jordan algebras, an important 
role is played by the \emph{Jordan triple product} $\{STW\}$ as defined by:

$ \{STW\} = (S \circ T)\circ W + (T \circ W) \circ S - (S \circ W) \circ T~, $
\med
which is linear in each factor and for which $\{STW\} = \{WTS\}$~. Certain examples entail 
setting $\{STW\} = \frac{1}{2}\{STW + WTS\}$~.

\med

 A \emph{Jordan Lie algebra} is a real vector space $\mathfrak A_{\bR}$
 together with a \emph{Jordan product} $\circ$ and \emph{Poisson bracket}
\bigbreak
 $\{~,~\}$, satisfying~:
 \begin{itemize}
\item[1.] for all $S, T \in \mathfrak A_{\bR}$,
$\begin{aligned} S \circ T &= T \circ S \\ \{S, T \} &= - \{T,
S\} \end{aligned}$
\med

\item[2.] the \emph{Leibniz rule} holds
\bigbreak
$ \{S, T \circ W \} = \{S, T\} \circ W + T \circ \{S, W\}$ 
for all $S, T, W \in \mathfrak A_{\bR}$, along with



\med

\item[3.]

the \emph{Jacobi identity}~:\\

 $ \{S, \{T, W \}\} = \{\{S,T \}, W\} + \{T, \{S, W \}\}$



\med

\item[4.]

for some $\hslash^2 \in \bR$, there is the \emph{associator identity} ~:
\bigbreak 
$(S \circ T) \circ W - S \circ (T \circ W) = \frac{1}{4} \hslash^2 \{\{S, W \}, T \}~.$



\end{itemize}



\subsubsection{Poisson algebra}


  By a \emph{Poisson algebra} we mean a Jordan algebra in which $\circ$ is associative. The 
usual algebraic types of morphisms automorphism, isomorphism, etc.) apply to Jordan-Lie 
(Poisson) algebras (see Landsman, 2003).


   Consider the classical configuration space $Q = \bR^3$ of a moving particle whose phase space 
is the cotangent bundle $T^* \bR^3 \cong \bR^6$, and for which the space of (classical) 
observables  is taken to be the real vector space of smooth functions

$$\mathfrak A^0_{\bR} = C^{\infty}(T^* R^3, \bR)$$~. The usual pointwise multiplication of 
functions $fg$ defines a bilinear map on $\mathfrak A^0_{\bR}$, which is seen to be 
commutative and associative. Further, the Poisson bracket on functions 

$$\{f, g \} := \frac{\del f}{\del p^i}  \frac{\del g}{\del q_i} - \frac{\del
f}{\del q_i} \frac{\del g}{\del p^i} ~,$$

 which can be easily seen to satisfy the Liebniz rule above. The axioms above then set the stage of passage to quantum mechanical systems which the parameter $k^2$ suggests.


\subsubsection{C*--algebras (C*--A), JLB and JBW Algebras} 

An \emph{involution} on a complex algebra $\mathfrak A$ is a real--linear map $T \mapsto T^*$ %%@
such that for all 
\bigbreak
$S, T \in \mathfrak A$ and $\lambda \in \bC$, we have $ T^{**} = T~,~ (ST)^* = T^* S^*~,~ %%@
(\lambda T)^* = \bar{\lambda} T^*~. $ 
\bigbreak
A \emph{*--algebra} is said to be a complex associative algebra together with an involution %%@
$*$~.


\med

A \emph{C*--algebra} is a simultaneously a *--algebra and a Banach space $\mathfrak A$, %%@
satisfying for all $S, T \in \mathfrak A$~:
\bigbreak

$ \begin{aligned} \Vert S \circ T \Vert &\leq \Vert S \Vert ~ \Vert T \Vert~, \\ \Vert T^* T %%@
\Vert^2 & = \Vert T\Vert^2 ~. \end{aligned}$
\bigbreak
We can easily see that $\Vert A^* \Vert = \Vert A \Vert$~. By the above axioms a C*--algebra %%@
is a special case of a Banach algebra where the latter requires the above norm property but %%@
not the involution (*) property. Given Banach spaces $E, F$ the space $\mathcal L(E, F)$ of %%@
(bounded) linear operators from $E$ to $F$
forms a Banach space, where for $E=F$, the space $\mathcal L(E) = \mathcal L(E, E)$ is a %%@
Banach algebra with respect to the norm \bigbreak
$\Vert T \Vert := \sup\{ \Vert Tu \Vert : u \in E~,~ \Vert u \Vert= 1 \}~. $
\bigbreak
In quantum field theory one may start with a Hilbert space $H$, and consider the Banach %%@
algebra of bounded linear operators $\mathcal L(H)$ which given to be closed under the usual %%@
algebraic operations and taking adjoints, forms a $*$--algebra of bounded operators, where the %%@
adjoint operation functions as the involution, and for $T \in \mathcal L(H)$ we have~: 
\bigbreak

$ \Vert T \Vert := \sup\{ ( Tu , Tu): u \in H~,~ (u,u) = 1 \}~,$ and $ \Vert Tu \Vert^2 = (Tu, %%@
Tu) = (u, T^*Tu) \leq \Vert T^* T \Vert~ \Vert u \Vert^2~.$
\bigbreak


By a morphism between C*--algebras $\mathfrak A,\mathfrak B$ we mean a linear map $\phi : %%@
\mathfrak A \lra \mathfrak B$, such that for all $S, T \in \mathfrak A$, the following hold~: 
\bigbreak
$\phi(ST) = \phi(S) \phi(T)~,~ \phi(T^*) = \phi(T)^*~, $ 
\bigbreak
where a bijective morphism  is said to be an isomorphism (in which case it is then an %%@
isometry). A fundamental relation is that any norm-closed $*$--algebra $\mathcal A$ in %%@
$\mathcal L(H)$ is a C*--algebra, and conversely, any C*--algebra is isomorphic to a %%@
norm--closed $*$--algebra in $\mathcal L(H)$ for some Hilbert space $H$~.

\med

For a C*--algebra $\mathfrak A$, we say that $T \in \mathfrak A$ is \emph{self--adjoint} if $T %%@
= T^*$~. Accordingly, the self--adjoint part $\mathfrak A^{sa}$ of $\mathfrak A$ is a real %%@
vector space since we can decompose $T \in \mathfrak A^{sa}$ as ~:
\bigbreak
$ T = T' + T^{''} := \frac{1}{2} (T + T^*) + \iota (\frac{-\iota}{2})(T - T^*)~.$
\bigbreak
 A \emph{commutative} C*--algebra is one for which the associative multiplication is %%@
commutative. Given a commutative C*--algebra $\mathfrak A$, we have $\mathfrak A \cong C(Y)$, %%@
the algebra of continuous functions on a compact Hausdorff space $Y~$.
 \med
A \emph{Jordan--Banach algebra} (a JB--algebra for short) is both a real Jordan algebra and a %%@
Banach space, where for all $S, T \in \mathfrak A_{\bR}$, we have 
\bigbreak
$ \begin{aligned} \Vert S \circ T \Vert &\leq \Vert S \Vert ~ \Vert T \Vert ~, \\ \Vert T %%@
\Vert^2 &\leq \Vert S^2 + T^2 \Vert ~. \end{aligned}$
\bigbreak
A \emph{JLB--algebra} is a JB--algebra $\mathfrak A_{\bR}$ together with a Poisson bracket for %%@
which it becomes a Jordan--Lie algebra for some $\hslash^2 \geq 0$~. Such  JLB--algebras often %%@
constitute the real part of several widely studied complex associative algebras.
\bigbreak
For the purpose of quantization, there are fundamental relations between $\mathfrak A^{sa}$, %%@
JLB and Poisson algebras. 
\med
%%In fact, if $\mathfrak A$ is a C*--algebra 
%%and $\hslash \in \bR/{0}$, then $\mathfrak A^{sa}$ is a JLB--algebra when it takes 
%%its norm from $\mathfrak A$, with $k= \hslash$, and is equipped with the operations~:
%%\bigbreak
%%$\begin{aligned} S \circ T &:= \frac{1}{2}{(ST + TS)} ~,{\left\{S,T\right\}}_k &:=\frac{\iota}_k \times [S,T]} ~.$ 
%%\end{aligned}$
%%\med
%%Conversely, given a JLB--algebra $\mathfrak A_{\bR}$ with $k^2 \geq 0$, its 
%%complexification $\mathfrak A$ is a $C^*$-algebra under the operations~: 
%%\bigbreak
%%$\begin{aligned} S T &:= S \circ T - \frac{\iota}{2} k \times \left\{S,T\right\}}_ ~, {(S + \iota T)}^* &:=S-\iota T %%%~. \end{aligned}$
\bigbreak
For further details see Landsman (2003) (Thm. 1.1.9).

\med
\bigbreak
A JB--algebra which is monotone complete and admits a separating set of normal sets is 
called a \emph{JBW-algebra}. These appeared in the work of von Neumann who developed a 
(orthomodular) lattice theory of projections on $\mathcal L(H)$ on which to study quantum 
logic (see later). BW-algebras have the following property: whereas $\mathfrak A^{sa}$ is a 
J(L)B--algebra, the self adjoint part of a von Neumann algebra is a JBW--algebra.

\bigbreak
\med

	A \emph{JC--algebra} is a norm closed real linear subspace of $\mathcal
L(H)^{sa}$ which is closed under the bilinear product 
$S \circ T = \frac{1}{2}(ST + TS)$ (non--commutative and nonassociative). Since any norm 
closed Jordan subalgebra of $\mathcal L(H)^{sa}$ is a JB--algebra, it is natural to specify 
the exact relationship between JB and JC--algebras, at least in finite dimensions. In order to 
do this, one introduces the `exceptional' algebra $H_3({\mathbb O})$, the algebra of $3 \times 
3$ Hermitian matrices with values in the octonians $\mathbb O$~. Then a finite dimensional JB--algebra is a 
JC--algebra if and only if it does not contain $H_3({\mathbb O})$ as a (direct) summand \cite{AS}. 


The above definitions and constructions follow the approach of Alfsen and Schultz (2003) and Landsman (1998).

\begin{thebibliography} {9}

\bibitem{AS}
Alfsen, E.M. and F. W. Schultz: Geometry of State Spaces of Operator Algebras, Birkh\"auser, Boston-Basel-Berlin.(2003). 

\end{thebibliography}

%%%%%
%%%%%
\end{document}
