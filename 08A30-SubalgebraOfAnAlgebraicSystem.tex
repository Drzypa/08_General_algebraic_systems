\documentclass[12pt]{article}
\usepackage{pmmeta}
\pmcanonicalname{SubalgebraOfAnAlgebraicSystem}
\pmcreated{2013-03-22 16:44:19}
\pmmodified{2013-03-22 16:44:19}
\pmowner{CWoo}{3771}
\pmmodifier{CWoo}{3771}
\pmtitle{subalgebra of an algebraic system}
\pmrecord{9}{38961}
\pmprivacy{1}
\pmauthor{CWoo}{3771}
\pmtype{Definition}
\pmcomment{trigger rebuild}
\pmclassification{msc}{08A30}
\pmclassification{msc}{08A05}
\pmclassification{msc}{08A62}
\pmsynonym{subalgebra lattice}{SubalgebraOfAnAlgebraicSystem}
\pmdefines{subalgebra}
\pmdefines{generating set}
\pmdefines{subalgebra generated by}
\pmdefines{extension of an algebraic system}
\pmdefines{restriction}
\pmdefines{proper subalgebra}
\pmdefines{lattice of subalgebras}
\pmdefines{spanning set}
\pmdefines{finitely generated}
\pmdefines{cyclic}

\usepackage{amssymb,amscd}
\usepackage{amsmath}
\usepackage{amsfonts}

% used for TeXing text within eps files
%\usepackage{psfrag}
% need this for including graphics (\includegraphics)
%\usepackage{graphicx}
% for neatly defining theorems and propositions
\usepackage{amsthm}
% making logically defined graphics
%%\usepackage{xypic}
\usepackage{pst-plot}
\usepackage{psfrag}

% define commands here
\newtheorem{prop}{Proposition}
\newtheorem{thm}{Theorem}
\newtheorem{ex}{Example}
\newcommand{\real}{\mathbb{R}}
\begin{document}
Let $(A,O)$ be an algebraic system ($A\ne \varnothing$ is the underlying set and $O$ is the set of operators on $A$).

\textbf{Subalgebras of an Algebra}

Let $B$ be a non-empty subset of $A$.  $B$ is \emph{closed} under operators of $A$ if for each $n$-ary operator $\omega_A$ on $A$, and any $b_1,\ldots,b_n\in B$, we have $\omega_A(b_1,\ldots,b_n)\in B$.

Suppose $B$ is closed under operators of $A$.  For each $n$-ary operator $\omega_A$ on $A$, we define $\omega_B:B^n\to B$ by $\omega_B(b_1,\ldots,b_n):= \omega_A(b_1,\ldots,b_n)$.  Each of these operators is well-defined and is called a \emph{restriction} (of the corresponding $\omega_A$).  Furthermore, $(B,O)$ is a well-defined algebraic system, and is called the \emph{subalgebra} of $(A,O)$.  When $(B,O)$ is a subalgebra of $(A,O)$, we also say that $(A,O)$ is an \emph{extension} of $(B,O)$.

$(A,O)$ is clearly a subalgebra of itself.  Any other subalgebra of $(A,O)$ is called a \emph{proper subalgebra}.

\textbf{Remark}.  If $(A,O)$ contains constants, then any subalgebra of $(A,O)$ must contain the exact same constants.  For example, the ring $\mathbb{Z}$ of integers is an algebraic system with no proper subalgebras.  Indeed, if $R$ is a subring of $\mathbb{Z}$, $1\in R$, so $R=\mathbb{Z}$.

Since we are operating under the same operator set, we can, for convenience, drop $O$ and simply call $A$ an algebra, $B$ a subalgebra of $A$, etc...  If $B_1,B_2$ are subalgebras of $A$, then $B_1\cap B_2$ is also a subalgebra.  In fact, given any set of subalgebras $B_i$ of $A$, their intersection $\bigcap B_i$ is also a subalgebra.

\textbf{Generating Set of an Algebra}

Let $C$ be any subset of an algebra $A$.  Consider the collection $[C]$ of all subalgebras of $A$ containing $C$.  This collection is non-empty because $A\in [C]$.  The intersection of all these subalgebras is again a subalgebra containing the set $C$.  Denote this subalgebra by $\langle C\rangle$.   $\langle C\rangle$ is called the subalgebra \emph{spanned} by $C$, and $C$ is called the \emph{spanning set} of $\langle C\rangle$.  Conversely, any subalgebra $B$ of $A$ has a spanning set, namely itself: $B=\langle B\rangle$.

Given a subalgebra $B$ of $A$, a minimal spanning set $X$ of $B$ is called a \emph{generating set} of $B$.  By minimal we mean that the set obtained by deleting any element from $X$ no longer spans $B$.  When $B$ has a generating set $X$, we also say that $X$ \emph{generates} $B$.   If $B$ can be generated by a finite set, we say that $B$ is \emph{finitely generated}.  If $B$ can be generated by a single element, we say that $B$ is \emph{cyclic}.

\textbf{Remark}.  $\langle \varnothing\rangle =$ the subalgebra generated by the constants of $A$.  If no such constants exist, $\langle \varnothing \rangle :=\varnothing$.

From the discussion above, the set of subalgebras of an algebraic system forms a complete lattice.  Given subalgebras $A_i$, $\bigvee A_i$ is the intersection of all $A_i$, and $\bigvee A_i$ is the subalgebra $\langle \bigcup A_i\rangle$.  The lattice of all subalgebras of $A$ is called the \emph{subalgebra latttice} of $A$, and is denoted by $\operatorname{Sub}(A)$.
%%%%%
%%%%%
\end{document}
