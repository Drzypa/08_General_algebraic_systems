\documentclass[12pt]{article}
\usepackage{pmmeta}
\pmcanonicalname{EnumeratingAlgebras}
\pmcreated{2013-03-22 15:50:48}
\pmmodified{2013-03-22 15:50:48}
\pmowner{Algeboy}{12884}
\pmmodifier{Algeboy}{12884}
\pmtitle{enumerating algebras}
\pmrecord{9}{37829}
\pmprivacy{1}
\pmauthor{Algeboy}{12884}
\pmtype{Theorem}
\pmcomment{trigger rebuild}
\pmclassification{msc}{08B99}
\pmclassification{msc}{05A16}
\pmrelated{EnumeratingGroups}

\endmetadata

\usepackage{latexsym}
\usepackage{amssymb}
\usepackage{amsmath}
\usepackage{amsfonts}
\usepackage{amsthm}

%%\usepackage{xypic}

%-----------------------------------------------------

%       Standard theoremlike environments.

%       Stolen directly from AMSLaTeX sample

%-----------------------------------------------------

%% \theoremstyle{plain} %% This is the default

\newtheorem{thm}{Theorem}

\newtheorem{coro}[thm]{Corollary}

\newtheorem{lem}[thm]{Lemma}

\newtheorem{lemma}[thm]{Lemma}

\newtheorem{prop}[thm]{Proposition}

\newtheorem{conjecture}[thm]{Conjecture}

\newtheorem{conj}[thm]{Conjecture}

\newtheorem{defn}[thm]{Definition}

\newtheorem{remark}[thm]{Remark}

\newtheorem{ex}[thm]{Example}



%\countstyle[equation]{thm}



%--------------------------------------------------

%       Item references.

%--------------------------------------------------


\newcommand{\exref}[1]{Example-\ref{#1}}

\newcommand{\thmref}[1]{Theorem-\ref{#1}}

\newcommand{\defref}[1]{Definition-\ref{#1}}

\newcommand{\eqnref}[1]{(\ref{#1})}

\newcommand{\secref}[1]{Section-\ref{#1}}

\newcommand{\lemref}[1]{Lemma-\ref{#1}}

\newcommand{\propref}[1]{Prop\-o\-si\-tion-\ref{#1}}

\newcommand{\corref}[1]{Cor\-ol\-lary-\ref{#1}}

\newcommand{\figref}[1]{Fig\-ure-\ref{#1}}

\newcommand{\conjref}[1]{Conjecture-\ref{#1}}


% Normal subgroup or equal.

\providecommand{\normaleq}{\unlhd}

% Normal subgroup.

\providecommand{\normal}{\lhd}

\providecommand{\rnormal}{\rhd}
% Divides, does not divide.

\providecommand{\divides}{\mid}

\providecommand{\ndivides}{\nmid}


\providecommand{\union}{\cup}

\providecommand{\bigunion}{\bigcup}

\providecommand{\intersect}{\cap}

\providecommand{\bigintersect}{\bigcap}
\begin{document}
\section{How many algebras are there?}

Unlike categories of discrete objects, such as simple graphs with $n$ vertices, (see \PMlinkname{article on enumerating graphs}{EnumeratingGraphs}) such a question is a little malposed as the quantity can be infinite.  However the spirit of the question can be addressed by appealing to algebraic varieties and considering their dimension.

Let $A$ be an non-associative algebra over a field $k$ of dimension $n$.  For example, $A$ could be a Lie algebra, an associative algebra, or a commutative algebra.  

From every basis $e_1,\dots, e_n$ for $A$, the addition of the algebra is completely understood as all $n$-dimensional $k$-vector spaces are isomorphic.  Thus we must consider only the multiplication.  For this the structure constants of the algebra are considered.  That is:
\[e_i e_j = \sum_{k=1}^n c^k_{ij} e_k\]
for $c^k_{ij}\in k$.  These structure constants completely define the algebra $A$.

Due to the axioms of multiplication, the structure constants satisfy certain relations.  For example, if $A$ is a Lie algebra then multiplication is via the associated Lie bracket and we know 
\[ [e_i,e_i]=0\]
Hence we find
\[c^k_{ii}=0\]
for all $1\leq i\leq n$.  Likewise the Jacobi identity/associativity/commutative conditions each imply their particular relations.  If one replaces the structure constants with variables $x_{ijk}$ we find that each algebra $A$ of a given type (Lie/Associative/Commutative/etc.) is a solution to the polynomial equations given by the relations of the algebra.  Thus the algebras themselves are parameterized by the algebraic variety, in $n^3$-dimensional affine space, of these equations.

\begin{thm}[Neretin, 1987]
The dimension of the algebraic variety for $n$-dimensional Lie algebras, associative
algebras, and commutative algebras is respectively
 \[\frac{2}{27}n^3 + O(n^{8/3}), \quad \frac{4}{27}n^3 + O(n^{8/3}),\]
 	\[\textnormal{ and }	\frac{2}{27}n^3 + O(n^{8/3}).\]
\end{thm}

Lower bounds of $\frac{2}{27}n^3+O(n^2)$ (and/or $\frac{4}{27}+O(n^2)$) are attainable by exhibiting large families of algebras.  For example, class 2 nilpotent Lie algebras attain the lower bound.  

As with the related problems for $p$-groups, it is also expected that the true upper bound has error term $O(n^2)$ [Neretin,Sims].


Neretin, Yu. A., \emph{An estimate for the number of parameters defining an
              {$n$}-dimensional algebra}, Izv. Akad. Nauk SSSR Ser. Mat., vol. 51,1987, no. 2, pp. 306--318, 447.\\

Mann, Avinoam, \emph{Some questions about {$p$}-groups},
  J. Austral. Math. Soc. Ser. A, vol. 67, 1999, no. 3, pp. 356--379.

%%%%%
%%%%%
\end{document}
