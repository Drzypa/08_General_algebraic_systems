\documentclass[12pt]{article}
\usepackage{pmmeta}
\pmcanonicalname{IsomorphismTheoremsOnAlgebraicSystems}
\pmcreated{2013-03-22 16:45:28}
\pmmodified{2013-03-22 16:45:28}
\pmowner{CWoo}{3771}
\pmmodifier{CWoo}{3771}
\pmtitle{isomorphism theorems on algebraic systems}
\pmrecord{8}{38984}
\pmprivacy{1}
\pmauthor{CWoo}{3771}
\pmtype{Theorem}
\pmcomment{trigger rebuild}
\pmclassification{msc}{08A05}

\usepackage{amssymb,amscd}
\usepackage{amsmath}
\usepackage{amsfonts}

% used for TeXing text within eps files
%\usepackage{psfrag}
% need this for including graphics (\includegraphics)
%\usepackage{graphicx}
% for neatly defining theorems and propositions
\usepackage{amsthm}
% making logically defined graphics
%%\usepackage{xypic}
\usepackage{pst-plot}
\usepackage{psfrag}

% define commands here
\newtheorem{prop}{Proposition}
\newtheorem{thm}{Theorem}
\newtheorem{ex}{Example}
\newcommand{\real}{\mathbb{R}}
\begin{document}
In this entry, all algebraic systems are of the same type; they are all $O$-algebras.  We list the generalizations of three famous isomorphism theorems, familiar to those who have studied abstract algebra in college.

\begin{thm}  If $f:A\to B$ is a homomorphism from algebras $A$ and $B$.  Then $$A/\ker(f)\cong f(A).$$
\end{thm}

\begin{thm}  If $B\subseteq A$ are algebras and $\mathfrak{C}$ is a \PMlinkname{congruence}{CongruenceRelationOnAnAlgebraicSystem} on $A$, then $$B/\mathfrak{C}_B\cong B^{\mathfrak{C}}/\mathfrak{C},$$
where $\mathfrak{C}_B$ is the congruence restricted to $B$, and $B^{\mathfrak{C}}$ is the extension of $B$ by $\mathfrak{C}$.
\end{thm}

\begin{thm}  If $A$ is an algebra and $\mathfrak{C}\subseteq \mathfrak{D}$ are congruences on $A$.  Then 
\begin{enumerate}
\item
there is a unique homomorphism $f:A/\mathfrak{C}\to A/\mathfrak{D}$ such that 
$$\xymatrix{
& A \ar[dl]_{[\cdot]_{\mathfrak{C}}} \ar[dr]^{[\cdot]_{\mathfrak{D}}} & \\
A/\mathfrak{C} \ar[rr]^f && A/\mathfrak{D}
}
$$
where the downward pointing arrows are the natural projections of $A$ onto the quotient algebras (induced by the respective congruences).  
\item
Furthermore, if $ker(f)=\mathfrak{D}/\mathfrak{C}$, then 
\begin{itemize}
\item
$\mathfrak{D}/\mathfrak{C}$ is a congruence on $A/\mathfrak{C}$, and 
\item
there is a unique isomorphism $f':A/\mathfrak{C} \to (A/\mathfrak{C})/(\mathfrak{D}/\mathfrak{C})$ satisfying the equation $f=[\cdot]_{\mathfrak{D}/\mathfrak{C}}\circ f'$.  In other words, 
$$(A/\mathfrak{C})/(\mathfrak{D}/\mathfrak{C})\cong A/\mathfrak{D}.$$
\end{itemize}
\end{enumerate}
\end{thm}
%%%%%
%%%%%
\end{document}
