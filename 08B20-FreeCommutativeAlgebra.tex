\documentclass[12pt]{article}
\usepackage{pmmeta}
\pmcanonicalname{FreeCommutativeAlgebra}
\pmcreated{2013-03-22 16:51:22}
\pmmodified{2013-03-22 16:51:22}
\pmowner{Algeboy}{12884}
\pmmodifier{Algeboy}{12884}
\pmtitle{free commutative algebra}
\pmrecord{5}{39103}
\pmprivacy{1}
\pmauthor{Algeboy}{12884}
\pmtype{Theorem}
\pmcomment{trigger rebuild}
\pmclassification{msc}{08B20}
\pmrelated{PolynomialRing}
\pmdefines{free commutative algebra}

\usepackage{latexsym}
\usepackage{amssymb}
\usepackage{amsmath}
\usepackage{amsfonts}
\usepackage{amsthm}

%%\usepackage{xypic}

%-----------------------------------------------------

%       Standard theoremlike environments.

%       Stolen directly from AMSLaTeX sample

%-----------------------------------------------------

%% \theoremstyle{plain} %% This is the default

\newtheorem{thm}{Theorem}

\newtheorem{coro}[thm]{Corollary}

\newtheorem{lem}[thm]{Lemma}

\newtheorem{lemma}[thm]{Lemma}

\newtheorem{prop}[thm]{Proposition}

\newtheorem{conjecture}[thm]{Conjecture}

\newtheorem{conj}[thm]{Conjecture}

\newtheorem{defn}[thm]{Definition}

\newtheorem{remark}[thm]{Remark}

\newtheorem{ex}[thm]{Example}



%\countstyle[equation]{thm}



%--------------------------------------------------

%       Item references.

%--------------------------------------------------


\newcommand{\exref}[1]{Example-\ref{#1}}

\newcommand{\thmref}[1]{Theorem-\ref{#1}}

\newcommand{\defref}[1]{Definition-\ref{#1}}

\newcommand{\eqnref}[1]{(\ref{#1})}

\newcommand{\secref}[1]{Section-\ref{#1}}

\newcommand{\lemref}[1]{Lemma-\ref{#1}}

\newcommand{\propref}[1]{Prop\-o\-si\-tion-\ref{#1}}

\newcommand{\corref}[1]{Cor\-ol\-lary-\ref{#1}}

\newcommand{\figref}[1]{Fig\-ure-\ref{#1}}

\newcommand{\conjref}[1]{Conjecture-\ref{#1}}


% Normal subgroup or equal.

\providecommand{\normaleq}{\unlhd}

% Normal subgroup.

\providecommand{\normal}{\lhd}

\providecommand{\rnormal}{\rhd}
% Divides, does not divide.

\providecommand{\divides}{\mid}

\providecommand{\ndivides}{\nmid}


\providecommand{\union}{\cup}

\providecommand{\bigunion}{\bigcup}

\providecommand{\intersect}{\cap}

\providecommand{\bigintersect}{\bigcap}










\begin{document}
Fix a commutative unital ring $K$ and a set $X$. Then a commutative associative $K$-algebra $F$ is said to be \emph{free on $X$} if there exists an injection $\iota:X\rightarrow F$ such that for all functions $f:X\rightarrow A$ where $A$ is a commutative $K$-algebra determine a unique algebra homomorphism $\hat{f}:F\rightarrow A$ such that $\iota\hat{f}=f$. This is an example of a universal mapping property for commutative associative algebras and in categorical settings is often explained with the following commutative diagram:
\[\xymatrix{
& X\ar[ld]_{\iota}\ar[rd]^{f} & \\
F\ar[rr]^{\hat{f}} & & A.
}\]
To construct a free commutative associative algebra we observe that commutative
associative algebras are a subcategory of associative algebras and thus we
can make use of free associative algebras in the construction and proof.

\begin{thm}
Given a set $X$, and a commutative unital ring $K$, the free commutative associative $K$-algebra on $X$ is the polynomial ring $K[X]$.
\end{thm}
\begin{proof}
Let $A$ be any commutative associative $K$-algebra and $f:X\rightarrow A$.
Recall $K\langle X\rangle$ is the free associative $K$-algebra on $X$ and
so by the universal mapping property of this free object there exists a 
map $\hat{f}:K\langle X\rangle \rightarrow A$ such that $\iota_{K\langle X\rangle} \hat{f}=f$.  

We also have a map $p:K\langle X\rangle\rightarrow K[X]$ which effectively
maps words over $X$ to words over $X$.  Only in $K[X]$ the 
indeterminants commute.  Since $A$ is commutative, $\hat{f}$ factors through
$p$, in the sense that there exists a map $\tilde{f}:K[X]\rightarrow A$ such
that $p\tilde{f}=\hat{f}$.  Thus $\tilde{f}$ is the desired map which proves $K[X]$ is free in the category of commutative
associative algebras.
\end{proof}

%%%%%
%%%%%
\end{document}
