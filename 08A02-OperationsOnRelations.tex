\documentclass[12pt]{article}
\usepackage{pmmeta}
\pmcanonicalname{OperationsOnRelations}
\pmcreated{2013-03-22 16:26:40}
\pmmodified{2013-03-22 16:26:40}
\pmowner{CWoo}{3771}
\pmmodifier{CWoo}{3771}
\pmtitle{operations on relations}
\pmrecord{14}{38600}
\pmprivacy{1}
\pmauthor{CWoo}{3771}
\pmtype{Definition}
\pmcomment{trigger rebuild}
\pmclassification{msc}{08A02}
\pmclassification{msc}{03E20}
\pmsynonym{inverse relation}{OperationsOnRelations}
\pmsynonym{relation composition}{OperationsOnRelations}
\pmsynonym{converse of a relation}{OperationsOnRelations}
\pmsynonym{diagonal relation}{OperationsOnRelations}
\pmrelated{GeneralAssociativity}
\pmrelated{RelationAlgebra}
\pmdefines{power of a relation}
\pmdefines{relational composition}
\pmdefines{inverse of a relation}
\pmdefines{section}
\pmdefines{domain}
\pmdefines{range}
\pmdefines{identity relation}

\endmetadata

\usepackage{amssymb,amscd}
\usepackage{amsmath}
\usepackage{amsfonts}

% used for TeXing text within eps files
%\usepackage{psfrag}
% need this for including graphics (\includegraphics)
%\usepackage{graphicx}
% for neatly defining theorems and propositions
%\usepackage{amsthm}
% making logically defined graphics
%%\usepackage{xypic}
\usepackage{pst-plot}
\usepackage{psfrag}

% define commands here

\begin{document}
\PMlinkescapeword{composition}
\PMlinkescapeword{inverse}
\PMlinkescapeword{converse}
\PMlinkescapeword{complement}
\PMlinkescapeword{power}
\PMlinkescapeword{identity}

Recall that an $n$-ary relation ($n>0$) is a subset of a product of some $n$ sets.  In this definition, any $n$-ary relation for which $n>1$ is automatically an $(n-1)$-ary relation, and consequently a binary relation.  On the other hand, a unary, or $1$-ary relation, being the subset $B$ of some set $A$, can be viewed as a binary relation (either realized as $B\times B$ or $\Delta_B:=\lbrace (b,b)\mid b\in B\rbrace$) on $A$.  Hence, we shall concentrate our discussion on the most important kind of relations, the binary relations, in this entry.

\subsubsection*{Compositions, Inverses, and Complements}

Let $\rho \subseteq A\times B$ and $\sigma \subseteq B\times C$ be two binary relations.  We define the \emph{composition} of $\rho$ and $\sigma$, the \emph{inverse} (or \emph{converse}) and the \emph{complement} of $\rho$ by
\begin{itemize}
\item $\rho\circ\sigma := \lbrace (a,c)\mid (a,b)\in \rho \mbox{ and }(b,c)\in \sigma \mbox{ for some } b\in B\rbrace$, 
\item $\rho^{-1}:=\lbrace (b,a)\mid (a,b)\in \rho\rbrace$, and
\item $\rho':=\lbrace (a,b)\mid (a,b)\notin \rho\rbrace$.
\end{itemize}
$\rho\circ\sigma,\rho^{-1}$ and $\rho'$ are binary relations of $A\times C, B\times A$ and $A\times B$ respectively.

\textbf{Properties}.
Suppose $\rho,\sigma,\tau$ are binary relations of $A\times B,B\times C,C\times D$ respectively.
\begin{enumerate}
\item Associativity of relational compositions: $(\rho\circ\sigma)\circ \tau =\rho\circ (\sigma\circ\tau)$.
\item $(\rho\circ\sigma)^{-1}=\sigma^{-1}\circ\rho^{-1}$.
\item For the rest of the remarks, we assume $A=B$.  Define the \emph{power} of $\rho$ recursively by $\rho^1=\rho$, and $\rho^{n+1}=\rho\circ\rho^n$.  By 1 above, $\rho^{n+m}=\rho^n\circ\rho^m$ for $m,n\in \mathbb{N}$.
\item $\rho^{-n}$ may be also be defined, in terms of $\rho^{-1}$ for $n>0$.
\item However, $\rho^{n+m}\ne\rho^n\circ\rho^m$ for \emph{all} integers, since $\rho^{-1}\circ\rho\neq \rho\circ\rho^{-1}$ in general.
\item Nevertheless, we may define $\Delta:=\lbrace (a,a)\mid a\in A\rbrace$.  This is called the \emph{identity} or \emph{diagonal relation} on $A$.  It has the property that $\Delta\circ \rho=\rho\circ\Delta=\rho$.  For this, we may define, for every relation $\rho$ on $A$, $\rho^0:=\Delta$.
\item Let $\mathcal{R}$ be the set of all binary relations on a set $A$.  Then $(\mathcal{R},\circ)$ is a \PMlinkname{monoid}{Monoid} with $\Delta$ as the identity element.
\end{enumerate}

Let $\rho$ be a binary relation on a set $A$, below are some special relations definable from $\rho$:
\begin{itemize}
\item
$\rho$ is reflexive if $\Delta\subseteq\rho$; 
\item 
$\rho$ is symmetric if $\rho=\rho^{-1}$; 
\item
$\rho$ is anti-symmetric if $\rho\cap\rho^{-1}\subseteq \Delta$; 
\item
$\rho$ is transitive if $\rho\circ\rho \subseteq \rho$;
\item
$\rho$ is a pre-order if it is reflexive and transitive  
\item
$\rho$ is a partial order if it is a pre-order and is anti-symmetric
\item
$\rho$ is an equivalence if it is a pre-order and is symmetric
\end{itemize}
Of these, only reflexivity is preserved by $\circ$ and both symmetry and anti-symmetry are preserved by the inverse operation.

\subsubsection*{Other Operations}

Some operations on binary relations yield unary relations.  The most common ones are the following:

Given a binary relation $\rho\in A\times B$, and an elements $a\in A$ and $b\in B$, define 
$$\rho(a,\cdot):=\lbrace y\mid (a,y)\in \rho\rbrace 
\quad\mbox{ and }\quad
\rho(\cdot,b):=\lbrace x\mid (x,b)\in \rho\rbrace.$$

$\rho(a,\cdot)\subseteq B$ is called the \emph{section of $\rho$ in $B$ based at $a$}, and $\rho(\cdot,b)\subseteq A$ is the \emph{section of $\rho$ in $A$ (based) at $b$}.  When the base points are not mentioned, we simply say a section of $\rho$ in $A$ or $B$, or an $A$-section or a $B$-section of $\rho$.

Finally, we define the domain and range of a binary relation $\rho\subseteq A\times B$, to be 
\begin{itemize}
\item $\operatorname{dom}(\rho):=\lbrace a\mid (a,b)\in \rho \mbox{ for some } b\in B\rbrace$, and 
\item $\operatorname{ran}(\rho):=\lbrace b\mid (a,b)\in \rho \mbox{ for some } a\in A\rbrace.$
\end{itemize}
respectively.  When $\rho$ is a function, the domain and range of $\rho$ coincide with the domain of range of $\rho$ as a function.

\textbf{Remarks}.  Given a binary relation $\rho\subseteq A\times B$.
\begin{enumerate}
\item $\rho(a,\cdot)$, realized as a binary relation $\lbrace a\rbrace \times \rho(a,\cdot)$ can be viewed as the composition of $\Delta_a\circ \rho$, where $\Delta_a=\lbrace (a,a)\rbrace$.  Similarly, $\rho\circ \Delta_b=\rho(\cdot,b) \times \lbrace b\rbrace$.
\item $\operatorname{dom}(\rho)$ is the disjoint union of $A$-sections of $\rho$ and $\operatorname{ran}(\rho)$ is the disjoint union of $B$-sections of $\rho$.
\item $\operatorname{dom}(\rho^{-1})=\operatorname{ran}(\rho)$ and $\operatorname{ran}(\rho^{-1})=\operatorname{dom}(\rho)$.
\item $\rho\circ\rho^{-1}=\operatorname{dom}(\rho)\times \operatorname{dom}(\rho)$ and $\rho^{-1}\circ\rho=\operatorname{ran}(\rho)\times \operatorname{ran}(\rho)$.
\item If $A=B$, then $\operatorname{dom}(\rho)\times A=\rho\circ A^2$ and $\operatorname{ran}(\rho) = A^2\circ \rho$.
\item The composition of binary relations can be generalized: let $R$ be a subset of $A_1\times \cdots \times A_n$ and $S$ be a subset of $B_1\times \cdots \times B_m$, where $m,n$ are positive integers.  Further, we assume that $A_n=B_1=C$.  Define $R\circ S$, as a subset of $A_1\times \cdots \times A_{n-1} \times B_2 \times \cdots B_m$, to be the set $$\lbrace (a_1, \ldots, a_{n-1}, b_2, \ldots, b_m) \mid \exists c\in C \mbox{ with }(a_1, \ldots, a_{n-1}, c)\in R\mbox{ and }(c, b_2, \ldots, b_m) \in S\rbrace.$$
If $m=n=2$, then we have the familiar composition of binary relations.  If $m=1$ and $n=2$, then $R\circ S=\lbrace a \mid (a,c)\in R\mbox{ and }c \in S\rbrace$.  The case where $m=2$ and $n=1$ is similar.  If $m=n=1$, then $R\circ S = \lbrace \mbox{ true }\rbrace$ if $R\cap S\ne \varnothing$.  Otherwise, it is set to $\lbrace \mbox{ false }\rbrace$.
\end{enumerate}
%%%%%
%%%%%
\end{document}
