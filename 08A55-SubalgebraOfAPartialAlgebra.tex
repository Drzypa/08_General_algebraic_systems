\documentclass[12pt]{article}
\usepackage{pmmeta}
\pmcanonicalname{SubalgebraOfAPartialAlgebra}
\pmcreated{2013-03-22 18:42:54}
\pmmodified{2013-03-22 18:42:54}
\pmowner{CWoo}{3771}
\pmmodifier{CWoo}{3771}
\pmtitle{subalgebra of a partial algebra}
\pmrecord{10}{41481}
\pmprivacy{1}
\pmauthor{CWoo}{3771}
\pmtype{Definition}
\pmcomment{trigger rebuild}
\pmclassification{msc}{08A55}
\pmclassification{msc}{03E99}
\pmclassification{msc}{08A62}
\pmdefines{weak subalgebra}
\pmdefines{relative subalgebra}
\pmdefines{subalgebra}

\usepackage{amssymb,amscd}
\usepackage{amsmath}
\usepackage{amsfonts}
\usepackage{mathrsfs}

% used for TeXing text within eps files
%\usepackage{psfrag}
% need this for including graphics (\includegraphics)
%\usepackage{graphicx}
% for neatly defining theorems and propositions
\usepackage{amsthm}
% making logically defined graphics
%%\usepackage{xypic}
\usepackage{pst-plot}

% define commands here
\newcommand*{\abs}[1]{\left\lvert #1\right\rvert}
\newtheorem{prop}{Proposition}
\newtheorem{thm}{Theorem}
\newtheorem{ex}{Example}
\newcommand{\real}{\mathbb{R}}
\newcommand{\pdiff}[2]{\frac{\partial #1}{\partial #2}}
\newcommand{\mpdiff}[3]{\frac{\partial^#1 #2}{\partial #3^#1}}
\begin{document}
Unlike an algebraic system, where there is only one way to define a subalgebra, there are several ways to define a subalgebra of a partial algebra.  

Suppose $\boldsymbol{A}$ and $\boldsymbol{B}$ are partial algebras of type $\tau$:
\begin{enumerate}
\item $\boldsymbol{B}$ is a \emph{weak subalgebra} of $\boldsymbol{A}$ if $B\subseteq A$, and $f_{\boldsymbol{B}}$ is a subfunction of $f_{\boldsymbol{A}}$ for every operator symbol $f\in \tau$.

In words, $\boldsymbol{B}$ is a weak subalgebra of $\boldsymbol{A}$ iff $B\subseteq A$, and for each $n$-ary symbol $f\in \tau$, if $b_1,\ldots, b_n \in B$ such that $f_B(b_1,\ldots, b_n)$ is defined, then $f_A(b_1,\ldots, b_n)$ is also defined, and is equal to $f_B(b_1, \ldots, b_n)$.
\item $\boldsymbol{B}$ is a \emph{relative subalgebra} of $\boldsymbol{A}$ if $B\subseteq A$, and $f_{\boldsymbol{B}}$ is a \PMlinkname{restriction of $f_{\boldsymbol{A}}$ relative to $B$}{Subfunction} for every operator symbol $f\in \tau$.

In words, $\boldsymbol{B}$ is a relative subalgebra of $\boldsymbol{A}$ iff $B\subseteq A$, and for each $n$-ary symbol $f\in \tau$, given $b_1,\ldots, b_n \in B$, $f_B(b_1,\ldots, b_n)$ is defined iff $f_A(b_1,\ldots, b_n)$ is and belongs to $B$, and they are equal.
\item $\boldsymbol{B}$ is a \emph{subalgebra} of $\boldsymbol{A}$ if $B\subseteq A$, and $f_{\boldsymbol{B}}$ is a \PMlinkname{restriction}{Subfunction} of $f_{\boldsymbol{A}}$ for every operator symbol $f\in \tau$.

In words, $\boldsymbol{B}$ is a subalgebra of $\boldsymbol{A}$ iff $B\subseteq A$, and for each $n$-ary symbol $f\in \tau$, given $b_1,\ldots, b_n \in B$, $f_B(b_1,\ldots, b_n)$ is defined iff $f_A(b_1,\ldots, b_n)$ is, and they are equal.
\end{enumerate}

Notice that if $\boldsymbol{B}$ is a weak subalgebra of $\boldsymbol{A}$, then every constant of $\boldsymbol{B}$ is  a constant of $\boldsymbol{A}$, and vice versa.

Every subalgebra is a relative subalgebra, and every relative subalgebra is a weak subalgebra.  But the converse is false for both statements.  Below are two examples.

\begin{enumerate}
\item  Let $F$ be a field.  Then every subalgebra of $F$ is a subfield, and every relative subalgebra of $F$ is a subring.
\item
Let $A$ be the set of all non-negative integers, and $-_A$ the ordinary subtraction on integers.  Consider the partial algebra $(A, -_A)$.
\begin{itemize}
\item Let $B=A$ and $-_B$ the usual subtraction on integers, but $x-_B y$ is only defined when $x,y\in B$ have the same parity.  Then $(B, -_B)$ is a weak subalgebra of $(A, -_A)$.  
\item Let $C$ be the set of all positive integers, and $-_C$ the ordinary subtraction.  Then $(C,-_C)$ is a relative subalgebra of $(A,-_A)$.  
\item Let $D$ be the set $\lbrace 0,1,\ldots, n\rbrace$ and $-_D$ the ordinary subtraction.  Then $(D,-_D)$ is a subalgebra of $(A,-_A)$.
\end{itemize}
Notice that $(B,-_B)$ is not a relative subalgebra of $(A,-_A)$, since $7-_B 6$ is not defined, even though $7 -A 6 = 1\in B$, and and $(C,-_C)$ is not a subalgebra of $(A,-_A)$, since $1 -_C 1$ is not defined in $C$, even though $1 -A 1$ is defined in $A$.
\end{enumerate}

\textbf{Remarks}.  
\begin{enumerate}
\item
A weak subalgebra $\boldsymbol{B}$ of $\boldsymbol{A}$ is a relative subalgebra iff given $b_1,\ldots, b_n\in B$ such that $f_A(b_1,\ldots, b_n)$ is defined and is in $B$, then $f_B(b_1,\ldots, b_n)$ is defined.  A relative subalgebra $\boldsymbol{B}$ of $\boldsymbol{A}$ is a subalgebra iff whenever $f_A(b_1,\ldots, b_n)$ is defined for $b_i\in B$, it is in $B$.
\item
Let $\boldsymbol{A}$ be a partial algebra of type $\tau$, and $B\subseteq A$.  For each $n$-ary function symbol $f\in \tau$, define $f_{\boldsymbol{B}}$ on $B$ as follows: $f_{\boldsymbol{B}}(b_1,\ldots, b_n)$ is defined in $B$ iff $f_{\boldsymbol{A}}(b_1,\ldots, b_n)$ is defined in $A$ and $f_{\boldsymbol{A}}(b_1,\ldots, b_n)\in B$.  This turns $\boldsymbol{B}$ into a partial algebra.  However, $\boldsymbol{B}$ may not be of type $\tau$, since $f_{\boldsymbol{B}}$ may not be defined at all on $B$.  When $\boldsymbol{B}$ is a partial algebra of type $\tau$, it is a relative subalgebra of $\boldsymbol{A}$.
\item
When $\boldsymbol{A}$ is an algebra, all three notions of subalgebras are equivalent (assuming that the partial operations on a weak subalgebra are all total).
\end{enumerate}

\begin{thebibliography}{7}
\bibitem{gg} G. Gr\"{a}tzer: {\em Universal Algebra}, 2nd Edition, Springer, New York (1978).
\end{thebibliography}
%%%%%
%%%%%
\end{document}
