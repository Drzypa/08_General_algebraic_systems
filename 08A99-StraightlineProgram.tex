\documentclass[12pt]{article}
\usepackage{pmmeta}
\pmcanonicalname{StraightlineProgram}
\pmcreated{2013-03-22 16:16:02}
\pmmodified{2013-03-22 16:16:02}
\pmowner{Algeboy}{12884}
\pmmodifier{Algeboy}{12884}
\pmtitle{straight-line program}
\pmrecord{10}{38377}
\pmprivacy{1}
\pmauthor{Algeboy}{12884}
\pmtype{Definition}
\pmcomment{trigger rebuild}
\pmclassification{msc}{08A99}
\pmclassification{msc}{20A05}
\pmclassification{msc}{20-00}
\pmrelated{word}
\pmrelated{Word}
\pmdefines{straight-line program}
\pmdefines{SLP}
\pmdefines{SLP representation}

\endmetadata

\usepackage{latexsym}
\usepackage{amssymb}
\usepackage{amsmath}
\usepackage{amsfonts}
\usepackage{amsthm}

%%\usepackage{xypic}

%-----------------------------------------------------

%       Standard theoremlike environments.

%       Stolen directly from AMSLaTeX sample

%-----------------------------------------------------

%% \theoremstyle{plain} %% This is the default

\newtheorem{thm}{Theorem}

\newtheorem{coro}[thm]{Corollary}

\newtheorem{lem}[thm]{Lemma}

\newtheorem{lemma}[thm]{Lemma}

\newtheorem{prop}[thm]{Proposition}

\newtheorem{conjecture}[thm]{Conjecture}

\newtheorem{conj}[thm]{Conjecture}

\newtheorem{defn}[thm]{Definition}

\newtheorem{remark}[thm]{Remark}

\newtheorem{ex}[thm]{Example}



%\countstyle[equation]{thm}



%--------------------------------------------------

%       Item references.

%--------------------------------------------------


\newcommand{\exref}[1]{Example-\ref{#1}}

\newcommand{\thmref}[1]{Theorem-\ref{#1}}

\newcommand{\defref}[1]{Definition-\ref{#1}}

\newcommand{\eqnref}[1]{(\ref{#1})}

\newcommand{\secref}[1]{Section-\ref{#1}}

\newcommand{\lemref}[1]{Lemma-\ref{#1}}

\newcommand{\propref}[1]{Prop\-o\-si\-tion-\ref{#1}}

\newcommand{\corref}[1]{Cor\-ol\-lary-\ref{#1}}

\newcommand{\figref}[1]{Fig\-ure-\ref{#1}}

\newcommand{\conjref}[1]{Conjecture-\ref{#1}}


% Normal subgroup or equal.

\providecommand{\normaleq}{\unlhd}

% Normal subgroup.

\providecommand{\normal}{\lhd}

\providecommand{\rnormal}{\rhd}
% Divides, does not divide.

\providecommand{\divides}{\mid}

\providecommand{\ndivides}{\nmid}


\providecommand{\union}{\cup}

\providecommand{\bigunion}{\bigcup}

\providecommand{\intersect}{\cap}

\providecommand{\bigintersect}{\bigcap}










\begin{document}
\begin{defn}
Given a set $S$, a \emph{straight-line program} (SLP) is a family of functions $\mathcal{F}=\{f_i:1\leq i\leq m\}$
\[f_{i}:S^{n+i}\rightarrow S\]
for some fixed $n\in\mathbb{N}$.  An SLP is \emph{evaluated} on a tuple
$(s_1,\dots,s_n)$ by recursion in the following way: $f_1$ is evaluated
on $(s_1,\dots,s_n)$ as a function.  The remaining evaluations are 
recursive.  So $f_2(s_1,\dots,s_n)$ denotes
\[f_2(s_1,\dots,s_n):=f_2(s_1,\dots,s_n,f_1(s_1,\dots,s_n)).\]
and in general 
\[f_{i+1}(s_1,\dots,s_n):=
f_{i+1}(s_1,\dots,s_n,f_1(s_1,\dots,s_n),\dots,f_i(s_1,\dots,s_n)).\]
The final output $f_m(s_1,\dots,s_n)$ is denoted $\mathcal{F}(s_1,\dots,s_n)$.
In this way we treat $\mathcal{F}$ as function from $S^n\rightarrow S$.
\end{defn}

SLPs arrise from the multiple meansings of expressions of the sort $a^n$
in a some algebraic structure $S$.  First of all, one can formally treat $a^n$ as the word $a\cdots a$ in $S$.  Secondly this can be interpreted as the actual result of this mulitplication.  

In the former meaning, actually storing a word of the form $a^n$ as 
$a\cdots a$ is difficult hence it is abreviated.  Other examples include
words such as $a^{10}(bc)^6$ where the values of $a,b,c$ are continually changing or even unknown.  Here an SLP can encode this word in such a way
that if we replace $a$ by $bc$, then the resulting new word would result
simply by evaluation the SLP at the input $(bc,b,c)$ instead of $(a,b,c)$.

In the second treatment where we whish to actually evaluate $a^n$ and the like,
we find the problem of understanding what $a^n$ means as a program.
Certainly we may have $a^4=(aa)(aa)=a(a(aa))$ etc.  However this equivalence 
neglects the problem of selecting a method of computing the result.  Usually
an efficient method is desired.   An SLP developed from simple functions
such as $f(x)=x^2$ and $f(x,y)=x+y$ formally address this problem.

The term \emph{straight-line} reflects the fact that evaluating an SLP can
be achieved by a program which does not branch or loop so its execution is
a straight-line.  It is common for SLPs to be built entirely from simple
functions such as $f(x,y)=x+y$ or $f(x)=x\times x$.

Because each element $f_i$ of an SLP is evaluated externally only on $s_1,\dots,s_n$ and the remaining inputs come internally from previous $f_j$,
$1\leq j\leq i$, it is convient to write definitions for $f_i$ as taking
inputs only from $(s_1,\dots,s_n)$ and implicitly allowing for the use of
the outputs of previous $f_j$'s.

SLPs can be defined in contexts other than semigroups including rings, modules, and polynomials.  Although they arise naturally to compress computations, they are also useful in describing smaller bounds for many combinatorial theorems related to algebraic objects.

It is possible for a function $f:S^n\rightarrow S$ to be defined equivalently by multiple SLPs and so a notion of equality of SLPs is stronger than equivalence of final outputs.

\begin{defn}
Two SLPs $\mathcal{F}=\{f_i:1\leq i\leq m\}$ and $\mathcal{G}=\{g_i:1\leq i\leq k\}$ are \emph{equivalent} if $\mathcal{F}$ and $\mathcal{G}$ can be evaluated
on the same inputs and for every input $(s_1,\dots,s_n)$, $\mathcal{F}(s_1,\dots,s_n)=\mathcal{G}(s_1,\dots,s_n)$.
When and SLP $\mathcal{F}$ is equivalent to a trivial SLP $\{f:S^n\rightarrow S\}$ we say that $\mathcal{F}$ is an \emph{SLP representation} of $f$.
\end{defn}

Every function $f:S^n\rightarrow S$ can be expressed as an SLP trivially by
$\mathcal{F}=\{f\}$.  However, this SLP is typically the least optimal for the actual evaluation of the output for a given input.  This leads to a hierarchy imposed on equivalent SLPs based on their associated computational length.


\begin{defn}
If an SLP represents an algebraic expression (alternatively a word in the generators) in the semigroup $S$ then the
\emph{computational length} or simply \emph{length} of the SLP is the maximum number of multiplications in the semigroup performed to evaluate the expression
using the SLP.
\end{defn}

It is evident that the trivial SLP of an algebraic expression has length equal to the length of the word.
%%%%%
%%%%%
\end{document}
