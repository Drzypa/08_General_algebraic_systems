\documentclass[12pt]{article}
\usepackage{pmmeta}
\pmcanonicalname{GroundedRelation}
\pmcreated{2013-03-22 17:48:38}
\pmmodified{2013-03-22 17:48:38}
\pmowner{Jon Awbrey}{15246}
\pmmodifier{Jon Awbrey}{15246}
\pmtitle{grounded relation}
\pmrecord{11}{40273}
\pmprivacy{1}
\pmauthor{Jon Awbrey}{15246}
\pmtype{Definition}
\pmcomment{trigger rebuild}
\pmclassification{msc}{08A70}
\pmclassification{msc}{08A02}
\pmclassification{msc}{03G15}
\pmclassification{msc}{03E20}
\pmclassification{msc}{03C05}
\pmclassification{msc}{03B10}
\pmrelated{Relation}

% this is the default PlanetMath preamble.  as your knowledge
% of TeX increases, you will probably want to edit this, but
% it should be fine as is for beginners.

% almost certainly you want these
\usepackage{amssymb}
\usepackage{amsmath}
\usepackage{amsfonts}

% used for TeXing text within eps files
%\usepackage{psfrag}
% need this for including graphics (\includegraphics)
%\usepackage{graphicx}
% for neatly defining theorems and propositions
%\usepackage{amsthm}
% making logically defined graphics
%%%\usepackage{xypic}

% there are many more packages, add them here as you need them

% define commands here

\begin{document}
A \textbf{grounded relation} over a sequence of sets is a mathematical object consisting of two components.  The first component is a subset of the cartesian product taken over the given sequence of sets, which sets are called the \textit{domains} of the relation.  The second component is just the cartesian product itself.

For example, if $L$ is a grounded relation over a finite sequence of sets, $X_1, \ldots, X_k$, then $L$ has the form $L = (F(L), G(L))$, where $F(L) \subseteq G(L) = X_1 \times \ldots \times X_k$.

\section{Remarks}

\begin{itemize}
\item
In various language that is used, $F(L)$ may be called the \textit{figure} or the \textit{graph} of $L$, while $G(L)$ may be called the \textit{ground} of $L$.
\item
The default assumption in almost all applications is that the domains of the grounded relation are nonempty sets, hence departures from this assumption need to be noted explicitly.
\item
In many applications all relations are considered relative to explicitly specified grounds.  In these settings it is conventional to refer to grounded relations somewhat more simply as ``relations".
\item
One often hears or reads the usage $``L \subseteq X_1 \times \ldots \times X_k"$ when the speaker or writer really means $``F(L) \subseteq X_1 \times \ldots \times X_k"$.  Be charitable in your interpretations.
\item
The cardinality of $G(L)$ is referred to as the \textit{adicity} or the \textit{arity} of the relation.  For example, in the finite case, $L$ may be described as $k$\textit{-adic} or $k$\textit{-ary}.
\item
The set $\mathrm{dom}_j(L) := X_j$ is referred to as the $j^{\mbox{\small{th}}}$ \textit{domain} of the relation.
\item
In the special case where $k = 2$, the set $X_1$ is called ``the domain" and the set $X_2$ is called ``the codomain" of the relation.
\end{itemize}

%%%%%
%%%%%
\end{document}
