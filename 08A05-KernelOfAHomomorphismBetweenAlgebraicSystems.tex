\documentclass[12pt]{article}
\usepackage{pmmeta}
\pmcanonicalname{KernelOfAHomomorphismBetweenAlgebraicSystems}
\pmcreated{2013-03-22 16:26:20}
\pmmodified{2013-03-22 16:26:20}
\pmowner{CWoo}{3771}
\pmmodifier{CWoo}{3771}
\pmtitle{kernel of a homomorphism between algebraic systems}
\pmrecord{11}{38593}
\pmprivacy{1}
\pmauthor{CWoo}{3771}
\pmtype{Definition}
\pmcomment{trigger rebuild}
\pmclassification{msc}{08A05}
\pmsynonym{induced congruence}{KernelOfAHomomorphismBetweenAlgebraicSystems}
\pmrelated{KernelOfAHomomorphismIsACongruence}
\pmrelated{KernelPair}
\pmdefines{congruence induced by a homomorphism}

\endmetadata

\usepackage{amssymb,amscd}
\usepackage{amsmath}
\usepackage{amsfonts}

% used for TeXing text within eps files
%\usepackage{psfrag}
% need this for including graphics (\includegraphics)
%\usepackage{graphicx}
% for neatly defining theorems and propositions
%\usepackage{amsthm}
% making logically defined graphics
%%\usepackage{xypic}
\usepackage{pst-plot}
\usepackage{psfrag}

% define commands here

\begin{document}
Let $f:(A,O)\to (B,O)$ be a homomorphism between two algebraic systems $A$ and $B$ (with $O$ as the operator set).  Each element $b\in B$ corresponds to a subset $K(b):=f^{-1}(b)$ in $A$.  Then $\lbrace K(b)\mid b\in B\rbrace$ forms a partition of $A$.  The \emph{kernel} $\ker(f)$ of $f$ is defined to be 
$$\ker(f):=\bigcup_{b\in B}K(b)\times K(b).$$

It is easy to see that $\ker(f)=\lbrace (x,y)\in A\times A\mid f(x)=f(y)\rbrace$.  Since it is a subset of $A\times A$, it is relation on $A$.  Furthermore, it is an equivalence relation on $A$:
{\footnote{In general, $\lbrace N_i\rbrace$ is a partition of a set $A$ iff $\bigcup N_i^2$ is an equivalence relation on $A$.}}

\begin{enumerate}
\item $\ker(f)$ is reflexive: for any $a\in A$, $a\in K(f(a))$, so that $(a,a)\in K(f(a))^2\subseteq \ker(f)$
\item $\ker(f)$ is symmetric: if $(a_1,a_2)\in \ker(f)$, then $f(a_1)=f(a_2)$, so that $(a_2,a_1)\in \ker(f)$
\item $\ker(f)$ is transitive: if $(a_1,a_2),(a_2,a_3)\in \ker(f)$, then $f(a_1)=f(a_2)=f(a_3)$, so $(a_1,a_3)\in \ker(f)$.
\end{enumerate}

We write $a_1 \equiv a_2 \pmod {\ker(f)}$ to denote $(a_1,a_2)\in \ker(f)$. 

In fact, $\ker(f)$ is a congruence relation: for any $n$-ary operator symbol $\omega\in O$, suppose $c_1,\ldots,c_n$ and $d_1,\ldots,d_n$ are two sets of elements in $A$ with $c_i\equiv d_i \mod \ker(f)$.  Then $$f(\omega_A(c_1,\ldots,c_n) = \omega_B(f(c_1),\ldots,f(c_n))=\omega_B(f(d_1),\ldots,f(d_n)) = f(\omega_A(d_1,\ldots,d_n)),$$ so $\omega_A(c_1,\ldots,c_n)\equiv \omega_A(d_1,\ldots,d_n) \pmod {\ker(f)}$.  For this reason, $\ker(f)$ is also called the \emph{congruence induced by} $f$.

\textbf{Example}.  If $A,B$ are groups and $f:A\to B$ is a group homomorphism.  Then the kernel of $f$, using the definition above is just the union of the square of the cosets of $$N=\lbrace x\mid f(x)=e\rbrace,$$ the traditional definition of the kernel of a group homomorphism (where $e$ is the identity of $B$).

\textbf{Remark}.  The above can be generalized.  See the \PMlinkname{analog}{KernelOfAHomomorphismIsACongruence} in model theory.
%%%%%
%%%%%
\end{document}
