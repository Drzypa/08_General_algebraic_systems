\documentclass[12pt]{article}
\usepackage{pmmeta}
\pmcanonicalname{CategoricalAlgebra}
\pmcreated{2013-03-22 18:13:27}
\pmmodified{2013-03-22 18:13:27}
\pmowner{bci1}{20947}
\pmmodifier{bci1}{20947}
\pmtitle{categorical algebra}
\pmrecord{79}{40810}
\pmprivacy{1}
\pmauthor{bci1}{20947}
\pmtype{Topic}
\pmcomment{trigger rebuild}
\pmclassification{msc}{08A99}
\pmclassification{msc}{08A05}
\pmclassification{msc}{08A70}
\pmsynonym{algebraic categories}{CategoricalAlgebra}
%\pmkeywords{representations}
%\pmkeywords{categorical algebra}
%\pmkeywords{algebra or algebraic representations}
%\pmkeywords{representable functors}
%\pmkeywords{algebraic categories}
%\pmkeywords{categories of algebraic structures}
%\pmkeywords{categories of logic algebras}
%\pmkeywords{universal-algebras}
%\pmkeywords{operator algebras}
\pmrelated{AlgebraicCategoryOfLMnLogicAlgebras}
\pmrelated{NonAbelianStructures}
\pmrelated{AbelianCategory}
\pmrelated{AxiomsForAnAbelianCategory}
\pmrelated{GeneralizedVanKampenTheoremsHigherDimensional}
\pmrelated{AxiomaticTheoryOfSupercategories}
\pmrelated{CategoricalOntology}
\pmrelated{NonCommutingGraphOfAGroup}
\pmrelated{NonAbelianStructures}
\pmdefines{algebraic representation}
\pmdefines{functor representation}
\pmdefines{representable functor}
\pmdefines{category of algebraic structures}
\pmdefines{category of logic algebras}

% this is the default PlanetMath preamble.
\usepackage{amssymb}
\usepackage{amsmath}
\usepackage{amsfonts}

% used for TeXing text within eps files
%\usepackage{psfrag}
% need this for including graphics (\includegraphics)
%\usepackage{graphicx}
% for neatly defining theorems and propositions
%\usepackage{amsthm}
% making logically defined graphics
%%%\usepackage{xypic}

% there are many more packages

% define commands here
\usepackage{amsmath, amssymb, amsfonts, amsthm, amscd, latexsym, enumerate}
\usepackage{xypic, xspace}
\usepackage[mathscr]{eucal}
\usepackage[dvips]{graphicx}
\usepackage[curve]{xy}

\setlength{\textwidth}{6.5in}
%\setlength{\textwidth}{16cm}
\setlength{\textheight}{9.0in}
%\setlength{\textheight}{24cm}

\hoffset=-.75in     %%ps format
%\hoffset=-1.0in     %%hp format
\voffset=-.4in


\theoremstyle{plain}
\newtheorem{lemma}{Lemma}[section]
\newtheorem{proposition}{Proposition}[section]
\newtheorem{theorem}{Theorem}[section]
\newtheorem{corollary}{Corollary}[section]

\theoremstyle{definition}
\newtheorem{definition}{Definition}[section]
\newtheorem{example}{Example}[section]
%\theoremstyle{remark}
\newtheorem{remark}{Remark}[section]
\newtheorem*{notation}{Notation}
\newtheorem*{claim}{Claim}

\renewcommand{\thefootnote}{\ensuremath{\fnsymbol{footnote}}}
\numberwithin{equation}{section}

\newcommand{\Ad}{{\rm Ad}}
\newcommand{\Aut}{{\rm Aut}}
\newcommand{\Cl}{{\rm Cl}}
\newcommand{\Co}{{\rm Co}}
\newcommand{\DES}{{\rm DES}}
\newcommand{\Diff}{{\rm Diff}}
\newcommand{\Dom}{{\rm Dom}}
\newcommand{\Hol}{{\rm Hol}}
\newcommand{\Mon}{{\rm Mon}}
\newcommand{\Hom}{{\rm Hom}}
\newcommand{\Ker}{{\rm Ker}}
\newcommand{\Ind}{{\rm Ind}}
\newcommand{\IM}{{\rm Im}}
\newcommand{\Is}{{\rm Is}}
\newcommand{\ID}{{\rm id}}
\newcommand{\grpL}{{\rm GL}}
\newcommand{\Iso}{{\rm Iso}}
\newcommand{\rO}{{\rm O}}
\newcommand{\Sem}{{\rm Sem}}
\newcommand{\SL}{{\rm Sl}}
\newcommand{\St}{{\rm St}}
\newcommand{\Sym}{{\rm Sym}}
\newcommand{\Symb}{{\rm Symb}}
\newcommand{\SU}{{\rm SU}}
\newcommand{\Tor}{{\rm Tor}}
\newcommand{\U}{{\rm U}}

\newcommand{\A}{\mathcal A}
\newcommand{\Ce}{\mathcal C}
\newcommand{\D}{\mathcal D}
\newcommand{\E}{\mathcal E}
\newcommand{\F}{\mathcal F}
%\newcommand{\grp}{\mathcal G}
\renewcommand{\H}{\mathcal H}
\renewcommand{\cL}{\mathcal L}
\newcommand{\Q}{\mathcal Q}
\newcommand{\R}{\mathcal R}
\newcommand{\cS}{\mathcal S}
\newcommand{\cU}{\mathcal U}
\newcommand{\W}{\mathcal W}

\newcommand{\bA}{\mathbb{A}}
\newcommand{\bB}{\mathbb{B}}
\newcommand{\bC}{\mathbb{C}}
\newcommand{\bD}{\mathbb{D}}
\newcommand{\bE}{\mathbb{E}}
\newcommand{\bF}{\mathbb{F}}
\newcommand{\bG}{\mathbb{G}}
\newcommand{\bK}{\mathbb{K}}
\newcommand{\bM}{\mathbb{M}}
\newcommand{\bN}{\mathbb{N}}
\newcommand{\bO}{\mathbb{O}}
\newcommand{\bP}{\mathbb{P}}
\newcommand{\bR}{\mathbb{R}}
\newcommand{\bV}{\mathbb{V}}
\newcommand{\bZ}{\mathbb{Z}}

\newcommand{\bfE}{\mathbf{E}}
\newcommand{\bfX}{\mathbf{X}}
\newcommand{\bfY}{\mathbf{Y}}
\newcommand{\bfZ}{\mathbf{Z}}

\renewcommand{\O}{\Omega}
\renewcommand{\o}{\omega}
\newcommand{\vp}{\varphi}
\newcommand{\vep}{\varepsilon}

\newcommand{\diag}{{\rm diag}}
\newcommand{\grp}{\mathcal G}
\newcommand{\dgrp}{{\mathsf{D}}}
\newcommand{\desp}{{\mathsf{D}^{\rm{es}}}}
\newcommand{\grpeod}{{\rm Geod}}
%\newcommand{\grpeod}{{\rm geod}}
\newcommand{\hgr}{{\mathsf{H}}}
\newcommand{\mgr}{{\mathsf{M}}}
\newcommand{\ob}{{\rm Ob}}
\newcommand{\obg}{{\rm Ob(\mathsf{G)}}}
\newcommand{\obgp}{{\rm Ob(\mathsf{G}')}}
\newcommand{\obh}{{\rm Ob(\mathsf{H})}}
\newcommand{\Osmooth}{{\Omega^{\infty}(X,*)}}
\newcommand{\grphomotop}{{\rho_2^{\square}}}
\newcommand{\grpcalp}{{\mathsf{G}(\mathcal P)}}

\newcommand{\rf}{{R_{\mathcal F}}}
\newcommand{\grplob}{{\rm glob}}
\newcommand{\loc}{{\rm loc}}
\newcommand{\TOP}{{\rm TOP}}

\newcommand{\wti}{\widetilde}
\newcommand{\what}{\widehat}

\renewcommand{\a}{\alpha}
\newcommand{\be}{\beta}
\newcommand{\grpa}{\grpamma}
%\newcommand{\grpa}{\grpamma}
\newcommand{\de}{\delta}
\newcommand{\del}{\partial}
\newcommand{\ka}{\kappa}
\newcommand{\si}{\sigma}
\newcommand{\ta}{\tau}

\newcommand{\med}{\medbreak}
\newcommand{\medn}{\medbreak \noindent}
\newcommand{\bign}{\bigbreak \noindent}

\newcommand{\lra}{{\longrightarrow}}
\newcommand{\ra}{{\rightarrow}}
\newcommand{\rat}{{\rightarrowtail}}
\newcommand{\ovset}[1]{\overset {#1}{\ra}}
\newcommand{\ovsetl}[1]{\overset {#1}{\lra}}
\newcommand{\hr}{{\hookrightarrow}}

\newcommand{\<}{{\langle}}

%\newcommand{\>}{{\rangle}}
%\usepackage{geometry, amsmath,amssymb,latexsym,enumerate}
%%%\usepackage{xypic}

\def\baselinestretch{1.1}

\hyphenation{prod-ucts}

%\grpeometry{textwidth= 16 cm, textheight=21 cm}

\newcommand{\sqdiagram}[9]{$$ \diagram  #1  \rto^{#2} \dto_{#4}&
#3  \dto^{#5} \\ #6    \rto_{#7}  &  #8   \enddiagram
\eqno{\mbox{#9}}$$ }

\def\C{C^{\ast}}

\newcommand{\labto}[1]{\stackrel{#1}{\longrightarrow}}

%\newenvironment{proof}{\noindent {\bf Proof} }{ \hfill $\Box$
%{\mbox{}}
\newcommand{\midsqn}[1]{\ar@{}[dr]|{#1}}
\newcommand{\quadr}[4]
{\begin{pmatrix} & #1& \\[-1.1ex] #2 & & #3\\[-1.1ex]& #4&
 \end{pmatrix}}
\def\D{\mathsf{D}}

\begin{document}
\subsection{Introduction: An Outline of Categorical Algebra}
This topic entry provides an outline of an important mathematical field called {\em categorical algebra}; although  specific definitions are in use for various applications of categorical algebra to specific algebraic structures, they do not cover the entire field. In the most general sense, \emph{categorical algebras}-- as introduced by Mac Lane in 1965 -- can be described as the study of representations of algebraic structures, either concrete or abstract, in terms of categories, functors and natural transformations. 

In a narrow sense, a \emph{categorical algebra} is an associative algebra, defined for any locally finite category and a commutative ring with unity. This notion may be considered as a generalization of both the concept of group algebra and that of an incidence algebra, much as the concept of category generalizes the notions of group and partially ordered set.


\subsection{Extensions of categorical algebra} 

\begin{itemize}
\item Thus, ultimately, since categories are interpretations of the \emph{axiomatic elementary theory of abstract categories (ETAC)}, so are categorical algebras. 

 The general definition of representation introduced above can be still further extended by introducing \emph{supercategorical algebras as interpretations of ETAS}, as explained next. 

\item Mac Lane (1976) wrote in his {\em Bull. AMS} review cited here as a verbatim quotation: 


 \emph{``On some occasions I have been tempted to try to define what algebra is,
can, or should be - most recently in concluding a survey [72] on Recent
advances in algebra. But no such formal definitions hold valid for long, since
algebra and its various subfields steadily change under the influence of ideas
and problems coming not just from logic and geometry, but from analysis,
other parts of mathematics, and extra mathematical sources. The progress of
mathematics does indeed depend on many interlocking, unexpected and
multiform developments.''}  
\end{itemize}

\subsection{Basic definitions}

 An \emph{algebraic representation} is generally defined as a \emph{morphism $\rho$ from an abstract algebraic structure  $\mathcal{A}_S$ to a concrete algebraic structure $A_c$}, a Hilbert space $\mathcal{H}$, or a family of linear operator spaces. 

The key notion of \PMlinkname{representable functor}{RepresentableFunctor} was first reported by Alexander Grothendieck in 1960.

\begin{definition}
Thus, when the latter concept is extended to categorical algebra, one has  a \emph{representable} functor $S$ from an arbitrary category $\mathcal{C}$ to the category of sets $Set$ if $S$ admits a \emph{functor representation} defined as follows. A \emph{functor representation of $S$} is defined as a pair, $({R}, \phi)$, which consists of an object $R$ of $\mathcal{C}$ and a family $\phi$ of equivalences $\phi (C): \Hom_{\mathcal{C}}(R,C) \cong S(C)$, which is natural in C, with C being any object in $\mathcal{C}$. When the functor $S$ has such a representation, it is also said to be \emph{represented by the object $R$} of $\mathcal{C}$. For each object $R$ of $\mathbf{C}$
one writes $h_{R}: \mathcal{C} \lra Set$ for the covariant $\Hom$--functor $h_{R}(C)\cong \Hom_{\mathcal{C}}(R,C)$. A \emph{representation} $(R, \phi)$ of ${S}$ is therefore \emph{a natural equivalence of functors}: 
\begin{equation}
\phi:  h_{R} \cong {S}~.
\end{equation}
\end{definition}

\begin{remark}
 The equivalence classes of such functor representations (defined as natural equivalences) determine directly an algebraic (\emph{groupoid}) structure.
\end{remark}

\subsection{Note:}
See also  in Expositions the entry about abstract and concrete algebras. 

\subsection{Application: Quantum Categories}

\begin{thebibliography}{9}

\bibitem{SML65}
Saunders Mac Lane: Categorical algebra., {\em Bull. AMS}, \textbf{71} (1965), 40-106., Zbl 0161.01601, MR 0171826, 

\bibitem{SML76}
Saunders Mac Lane: Topology and Logic as a Source of Algebras., {\em Bull. AMS}, \textbf{82}, Number 1, 1-36,
January 1, 1976.

\bibitem{WK2k10}
\PMlinkexternal{Categorical algebra- basic definitions}{http://en.wikipedia.org/wiki/Categorical_algebra}

\end{thebibliography}


%%%%%
%%%%%
\end{document}
