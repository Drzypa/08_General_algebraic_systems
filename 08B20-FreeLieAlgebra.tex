\documentclass[12pt]{article}
\usepackage{pmmeta}
\pmcanonicalname{FreeLieAlgebra}
\pmcreated{2013-03-22 16:51:11}
\pmmodified{2013-03-22 16:51:11}
\pmowner{Algeboy}{12884}
\pmmodifier{Algeboy}{12884}
\pmtitle{free Lie algebra}
\pmrecord{5}{39099}
\pmprivacy{1}
\pmauthor{Algeboy}{12884}
\pmtype{Definition}
\pmcomment{trigger rebuild}
\pmclassification{msc}{08B20}
\pmrelated{LieAlgebra}
\pmrelated{UniversalEnvelopingAlgebra}
\pmrelated{PoincareBirkhoffWittTheorem}
\pmdefines{free Lie algebra}

\endmetadata

\usepackage{latexsym}
\usepackage{amssymb}
\usepackage{amsmath}
\usepackage{amsfonts}
\usepackage{amsthm}

%%\usepackage{xypic}

%-----------------------------------------------------

%       Standard theoremlike environments.

%       Stolen directly from AMSLaTeX sample

%-----------------------------------------------------

%% \theoremstyle{plain} %% This is the default

\newtheorem{thm}{Theorem}

\newtheorem{coro}[thm]{Corollary}

\newtheorem{lem}[thm]{Lemma}

\newtheorem{lemma}[thm]{Lemma}

\newtheorem{prop}[thm]{Proposition}

\newtheorem{conjecture}[thm]{Conjecture}

\newtheorem{conj}[thm]{Conjecture}

\newtheorem{defn}[thm]{Definition}

\newtheorem{remark}[thm]{Remark}

\newtheorem{ex}[thm]{Example}



%\countstyle[equation]{thm}



%--------------------------------------------------

%       Item references.

%--------------------------------------------------


\newcommand{\exref}[1]{Example-\ref{#1}}

\newcommand{\thmref}[1]{Theorem-\ref{#1}}

\newcommand{\defref}[1]{Definition-\ref{#1}}

\newcommand{\eqnref}[1]{(\ref{#1})}

\newcommand{\secref}[1]{Section-\ref{#1}}

\newcommand{\lemref}[1]{Lemma-\ref{#1}}

\newcommand{\propref}[1]{Prop\-o\-si\-tion-\ref{#1}}

\newcommand{\corref}[1]{Cor\-ol\-lary-\ref{#1}}

\newcommand{\figref}[1]{Fig\-ure-\ref{#1}}

\newcommand{\conjref}[1]{Conjecture-\ref{#1}}


% Normal subgroup or equal.

\providecommand{\normaleq}{\unlhd}

% Normal subgroup.

\providecommand{\normal}{\lhd}

\providecommand{\rnormal}{\rhd}
% Divides, does not divide.

\providecommand{\divides}{\mid}

\providecommand{\ndivides}{\nmid}


\providecommand{\union}{\cup}

\providecommand{\bigunion}{\bigcup}

\providecommand{\intersect}{\cap}

\providecommand{\bigintersect}{\bigcap}










\begin{document}
Fix a set $X$ and a commuative unital ring $K$.  A free $K$-Lie algebra $\mathfrak{L}$ on $X$ is any Lie algebra together with an injection $\iota:X\rightarrow \mathfrak{L}$ 
such that for any $K$-Lie algebra $\mathfrak{g}$ and function 
$f:X\rightarrow \mathfrak{g}$ implies
the existance of a unique Lie algebra homomorphism $\hat{f}:\mathfrak{L}\rightarrow \mathfrak{g}$ 
where $\iota\hat{f}=f$.  This universal mapping property is commonly expressed
as a commutative diagram:
\[\xymatrix{
 & X\ar[ld]_{\iota}\ar[rd]^{f} & \\
\mathfrak{L}\ar[rr]^{\hat{f}} & & \mathfrak{g}.
}\]

To construct a free Lie algebra is generally and indirect process.
We begin with any free associative algebra $K\langle X\rangle$ on $X$, 
which can be constructed as the tensor algebra over a free $K$-module 
with basis $X$.  Then $K\langle X\rangle^-$ is a $K$-Lie algebra with the
standard commutator bracket $[a,b]=ab-ba$ for $a,b\in K\langle X\rangle$.

Now define $\mathfrak{FL}_K\langle X\rangle$ as the Lie subalgebra of $K\langle X\rangle^-$
generated by $X$.

\begin{thm}[Witt]\cite[Thm V.7]{Jacobson}
$\mathfrak{FL}_K\langle X\rangle$ is a free Lie algebra on $X$ and its universal 
enveloping algebra is $K\langle X\rangle$.
\end{thm}

It is generally not true that $\mathfrak{FL}_K\langle X\rangle=K\langle X\rangle^-$.  For example, if $x\in X$ then $x^2\in K\langle X\rangle$ but 
$x^2$ is not in $\mathfrak{FL}_K\langle X\rangle$.

\bibliographystyle{amsplain}
\begin{thebibliography}{10}
\bibitem{Jacobson}
Nathan Jacobson \emph{Lie Algebras}, Interscience Publishers, New York, 1962.

\end{thebibliography}

%%%%%
%%%%%
\end{document}
