\documentclass[12pt]{article}
\usepackage{pmmeta}
\pmcanonicalname{GroundFieldsAndRings}
\pmcreated{2013-03-22 15:54:22}
\pmmodified{2013-03-22 15:54:22}
\pmowner{Algeboy}{12884}
\pmmodifier{Algeboy}{12884}
\pmtitle{ground fields and rings}
\pmrecord{15}{37909}
\pmprivacy{1}
\pmauthor{Algeboy}{12884}
\pmtype{Definition}
\pmcomment{trigger rebuild}
\pmclassification{msc}{08A30}
%\pmkeywords{ground ring}
\pmrelated{ExtensionField}
\pmrelated{FieldAdjunction}
\pmrelated{RingAdjunction}
\pmdefines{ground field}
\pmdefines{base field}
\pmdefines{ground ring}
\pmdefines{base ring}

\endmetadata

\usepackage{latexsym}
\usepackage{amssymb}
\usepackage{amsmath}
\usepackage{amsfonts}
\usepackage{amsthm}

%%\usepackage{xypic}

%-----------------------------------------------------

%       Standard theoremlike environments.

%       Stolen directly from AMSLaTeX sample

%-----------------------------------------------------

%% \theoremstyle{plain} %% This is the default

\newtheorem{thm}{Theorem}

\newtheorem{coro}[thm]{Corollary}

\newtheorem{lem}[thm]{Lemma}

\newtheorem{lemma}[thm]{Lemma}

\newtheorem{prop}[thm]{Proposition}

\newtheorem{conjecture}[thm]{Conjecture}

\newtheorem{conj}[thm]{Conjecture}

\newtheorem{defn}[thm]{Definition}

\newtheorem{remark}[thm]{Remark}

\newtheorem{ex}[thm]{Example}



%\countstyle[equation]{thm}



%--------------------------------------------------

%       Item references.

%--------------------------------------------------


\newcommand{\exref}[1]{Example-\ref{#1}}

\newcommand{\thmref}[1]{Theorem-\ref{#1}}

\newcommand{\defref}[1]{Definition-\ref{#1}}

\newcommand{\eqnref}[1]{(\ref{#1})}

\newcommand{\secref}[1]{Section-\ref{#1}}

\newcommand{\lemref}[1]{Lemma-\ref{#1}}

\newcommand{\propref}[1]{Prop\-o\-si\-tion-\ref{#1}}

\newcommand{\corref}[1]{Cor\-ol\-lary-\ref{#1}}

\newcommand{\figref}[1]{Fig\-ure-\ref{#1}}

\newcommand{\conjref}[1]{Conjecture-\ref{#1}}


% Normal subgroup or equal.

\providecommand{\normaleq}{\unlhd}

% Normal subgroup.

\providecommand{\normal}{\lhd}

\providecommand{\rnormal}{\rhd}
% Divides, does not divide.

\providecommand{\divides}{\mid}

\providecommand{\ndivides}{\nmid}


\providecommand{\union}{\cup}

\providecommand{\bigunion}{\bigcup}

\providecommand{\intersect}{\cap}

\providecommand{\bigintersect}{\bigcap}










\begin{document}
The following is a list of common uses of the \PMlinkescapetext{term} \emph{ground} or \emph{base} field or ring in algebra.  These \PMlinkescapetext{terms} are endowed with \PMlinkescapetext{semantics} based on their context so the following list may be \PMlinkescapetext{incomplete} or may not apply uniformly.

One commonality is generally found for the use of ground ring or field: the result is a unitial subring of the original.  Outside of this requirement, the constraints are specific to context.

\begin{itemize}
\item  Given a ring $R$ with a 1, let $\mathbb{Z}1$ be the subgroup of $R$ generated by $1$ under addition.  This is consequently a subring of $R$ of the same characteristic as $R$.  Thus is it isomorphic to $\mathbb{Z}/c\mathbb{Z}$ where $c$ is the characteristic of $R$.  This is the smallest unital subring of $R$ and so rightfully may be called the ground or base ring of $R$.

When the characteristic of $R$ is prime, $\mathbb{Z}1\cong \mathbb{Z}/p\mathbb{Z}$ and so it may be called the ground field of $R$.

\item Given a vector space or algebra $A$ over a field $k$, then $k$ is the ground/base field of $A$.  

\item Given a set of matrices $M_n(R)$, the ground ring is commonly the ring $R$, and if required as a subring of $M_n(R)$ then it is taken as the set of all scalar matrices.

\item Given a field extension $K/k$ over a field $k$, then $k$ is the ground field of $K$ in this context.  For a general field where no specific subfield has been specified, the ground/base field then typically defaults to the prime subfield of $K$.  (Recall the prime subfield is the unique smallest subfield of $K$.)

\item Given a field $K$ and a set of field automorphisms\, $f:K\rightarrow K$,\, the ground/base field in this context is the \PMlinkname{fixed field}{Fixed} of the automorphisms.  That is, the largest subfield of $K$ which is pointwise fixed by each $f$.  Since a field automorphism must fix the prime subfield, this definition always produces a field containing the prime subfield.

\end{itemize}

%%%%%
%%%%%
\end{document}
