\documentclass[12pt]{article}
\usepackage{pmmeta}
\pmcanonicalname{Biops}
\pmcreated{2013-03-22 14:44:49}
\pmmodified{2013-03-22 14:44:49}
\pmowner{HkBst}{6197}
\pmmodifier{HkBst}{6197}
\pmtitle{biops}
\pmrecord{7}{36385}
\pmprivacy{1}
\pmauthor{HkBst}{6197}
\pmtype{Definition}
\pmcomment{trigger rebuild}
\pmclassification{msc}{08A99}
\pmdefines{semigroup}
\pmdefines{monoid}
\pmdefines{group}
\pmdefines{rig}
\pmdefines{ring}
\pmdefines{quasigroup}
\pmdefines{loop}

% this is the default PlanetMath preamble.  as your knowledge
% of TeX increases, you will probably want to edit this, but
% it should be fine as is for beginners.

% almost certainly you want these
\usepackage{amssymb}
\usepackage{amsmath}
\usepackage{amsfonts}

% used for TeXing text within eps files
%\usepackage{psfrag}
% need this for including graphics (\includegraphics)
%\usepackage{graphicx}
% for neatly defining theorems and propositions
%\usepackage{amsthm}
% making logically defined graphics
%%%\usepackage{xypic}

% there are many more packages, add them here as you need them

% define commands here
\begin{document}
Let $S$ be a set and $n \in \mathbf{N}$. Set $\mathbf{N}_n := \{i \in \mathbf{N} | i < n \}$. If there exists a map $\cdot : \mathbf{N}_n \to (S^2 \to S) : i \mapsto \cdot_i$ where $\cdot_i : S^2 \to S : (a, b) \mapsto a \cdot_i b$ is a binary operation, then I shall say that $(S, \cdot)$ is an \emph{$n$-biops}.  In other words, an $n$-biops is an algebraic system with $n$ binary operations defined on it, and the operations are labelled $0,1,\ldots, n-1$.

Let $(S, \cdot)$ be an $n$-biops. If $\cdot$ has the property $p$, then I shall say that $(S, \cdot)$ is a $p$ $n$-biops. 

For example if $(S, \cdot)$ is an $n$-biops and $\cdot$ is $0$-commutative, $0$-associative, $0$-alternative or $(0, 1)$-distributive, then I shall say that $(S, \cdot)$ is a $0$-commutative $n$-biops, $0$-associative $n$-biops, $0$-alternative $n$-biops or $(0, 1)$-distributive $n$-biops respectively.

If an $n$-biops $B$ is $i$-$p$ for each $i \in \mathbf{N}_n$ then I shall say that $B$ is a $p$ $n$-biops.

A $0$-associative $1$-biops is called a semigroup.
A semigroup with identity element is called a monoid.
A monoid with inverses is called a group.

A $(0, 1)$-distributive $2$-biops $(S, +, \cdot)$, such that both $(S, +)$ and $(S, \cdot)$ are monoids, is called a rig.

A $(0, 1)$-distributive $2$-biops $(S, +, \cdot)$, such that $(S, +)$ is a group and $(S, \cdot)$ is a monoid, is called a ring.

A rig with $0$-inverses is a ring. 

A $0$-associative $2$-biops $(S, \cdot, /)$ with $0$-identity such that for every $\{a, b\} \subset S$ we have
$$b = (b / a) \cdot a = (b \cdot a) / a$$
is called a group.

A $3$-biops $(S, \cdot, /, \backslash)$ such that for every $\{a, b\} \subset S$ we have
$$a \backslash (a \cdot b) = a \cdot (a \backslash b) = b = (b / a) \cdot a = (b \cdot a) / a$$
is called a quasigroup.

A quasigroup such that for every $\{a, b\} \subset S$ we have $a / a = b \backslash b$  is called a loop.

A $0$-associative loop is a group.
%%%%%
%%%%%
\end{document}
