\documentclass[12pt]{article}
\usepackage{pmmeta}
\pmcanonicalname{RelationalSystem}
\pmcreated{2013-03-22 16:35:33}
\pmmodified{2013-03-22 16:35:33}
\pmowner{CWoo}{3771}
\pmmodifier{CWoo}{3771}
\pmtitle{relational system}
\pmrecord{16}{38788}
\pmprivacy{1}
\pmauthor{CWoo}{3771}
\pmtype{Definition}
\pmcomment{trigger rebuild}
\pmclassification{msc}{08A55}
\pmclassification{msc}{03C07}
\pmclassification{msc}{08A02}
\pmsynonym{relational structure}{RelationalSystem}
\pmrelated{AlgebraicSystem}
\pmrelated{PartialAlgebraicSystem}
\pmrelated{Structure}
\pmrelated{StructuresAndSatisfaction}

\usepackage{amssymb,amscd}
\usepackage{amsmath}
\usepackage{amsfonts}

% used for TeXing text within eps files
%\usepackage{psfrag}
% need this for including graphics (\includegraphics)
%\usepackage{graphicx}
% for neatly defining theorems and propositions
%\usepackage{amsthm}
% making logically defined graphics
%%\usepackage{xypic}
\usepackage{pst-plot}
\usepackage{psfrag}

% define commands here

\begin{document}
A \emph{relational system}, loosely speaking, is a pair $(A,R)$ where $A$ is a set and $R$ is a set of finitary relations defined on $A$ (a finitary relation is just an $n$-ary relation where $n\in\mathbb{N}$; when $n=1$, it is called a property).  Since an $n$-ary operator on a set is an $(n+1)$-ary \PMlinkescapetext{relation on} the set, a relational system can be thought of as a generalization of an algebraic system.  We can formalize the notion of a relation system as follows:
\begin{quote}
Call a set $R$ a relation set, if there is a function $f:R\to \mathbb{N}$, the set of natural numbers.  For each $r\in R$, call $f(r)$ the arity of $r$.

Let $A$ be a set and $R$ a \emph{relation set}.  The pair $(A,R)$ is called an $R$-relational system if there is a set $R_A$ such that
\begin{itemize}
\item $R_A$ is a set of finitary relations on $A$, called the \emph{relation set} of $A$, and
\item there is a one-to-one correspondence between $R$ and $R_A$, given by $r \mapsto r_A$, such that the $f(r)=$ the arity of $r_A$.
\end{itemize}
\end{quote}

Since operators and partial operators are special types of relations.  algebraic systems and partial algebraic systems can be treated as relational systems.

Below are some exmamples of relational systems:
\begin{itemize}
\item any algebraic or partial algebraic system.
\item a poset $(P,\lbrace \le_P\rbrace)$, where $\le_P$ is a binary relation, called the partial ordering, on $P$.  A lattice, generally considered an algebraic system, can also be considered as a relational system, because it is a poset, and that $\le$ alone defines the algebraic operations ($\vee$ and $\wedge$).
\item a pointed set $(A,\lbrace a\rbrace)$ is also a relational system, where a unary relation, or property, is the singled-out element $a\in A$.  A pointed set is also an algebraic system, if we treat $a$ as the lone nullary operator (constant).
\item a bounded poset $(P,\le_P,0,1)$ is a relational system.  It is a poset, with two unary relations $\lbrace 0\rbrace$ and $\lbrace 1\rbrace$.
\item a Buekenhout-Tits geometry can be thought of as a relational system.  It consists of a set $\Gamma$ with two binary relations on it: one is an equivalence relation $T$ called type, and the other is a symmetric reflexive relation $\#$ called incidence, such that if $a\#b$ and $aTb$, then $a=b$ (incident objects of the same type are identical).
\item ordered algebraic structures, such as ordered groups $(G,\lbrace \cdot\mbox{, }^{-1}\mbox{, }e\mbox{, }\le_G\rbrace)$ and ordered rings $(R,\lbrace +\mbox{, }-\mbox{, }\cdot\mbox{, }^{-1}\mbox{, }0\mbox{, }\le_R\rbrace)$ are also relational systems.  They are not algebraic systems because of the additional ordering relations ($\le_G$ and $\le_R$) defined on these objects.  Note that these orderings are generally considered total orders.
\item ordered partial algebras such as ordered fields $(D,\lbrace +\mbox{, }-\mbox{, }\cdot\mbox{, }^{-1}\mbox{, }0\mbox{, }1\mbox{, }\le_F\rbrace)$, etc...  
\item structures that are not relational are \PMlinkname{complete lattices}{CompleteLattice} and topological spaces, because the operations involved are infinitary.
\end{itemize}

\textbf{Remark}.  Relational systems and algebraic systems are both examples of structures in model theory.  Although an algebraic system is a relational system in the sense discussed above, they are treated as distinct entities.   A structure involves three objects, a set $A$, a set of function symbols $F$, and a set of relation symbols $R$, so a relational system is a structure where $F=\varnothing$ and an algebraic system is a structure where $R=\varnothing$.

\begin{thebibliography}{7}
\bibitem{gg} G. Gr\"{a}tzer: {\em Universal Algebra}, 2nd Edition, Springer, New York (1978).
\end{thebibliography}
%%%%%
%%%%%
\end{document}
