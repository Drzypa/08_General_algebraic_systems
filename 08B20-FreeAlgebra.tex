\documentclass[12pt]{article}
\usepackage{pmmeta}
\pmcanonicalname{FreeAlgebra}
\pmcreated{2013-03-22 16:51:05}
\pmmodified{2013-03-22 16:51:05}
\pmowner{CWoo}{3771}
\pmmodifier{CWoo}{3771}
\pmtitle{free algebra}
\pmrecord{6}{39097}
\pmprivacy{1}
\pmauthor{CWoo}{3771}
\pmtype{Definition}
\pmcomment{trigger rebuild}
\pmclassification{msc}{08B20}
\pmsynonym{free algebraic system}{FreeAlgebra}
\pmrelated{TermAlgebra}
\pmdefines{free generating set}

\endmetadata

\usepackage{amssymb,amscd}
\usepackage{amsmath}
\usepackage{amsfonts}

% used for TeXing text within eps files
%\usepackage{psfrag}
% need this for including graphics (\includegraphics)
%\usepackage{graphicx}
% for neatly defining theorems and propositions
\usepackage{amsthm}
% making logically defined graphics
%%\usepackage{xypic}
\usepackage{pst-plot}
\usepackage{psfrag}

% define commands here
\newtheorem{prop}{Proposition}
\newtheorem{thm}{Theorem}
\newtheorem{ex}{Example}
\newcommand{\real}{\mathbb{R}}
\begin{document}
Let $\mathcal{K}$ be a class of algebraic systems (of the same type $\tau$).  Consider an algebra $A\in \mathcal{K}$ \PMlinkname{generated by}{SubalgebraOfAnAlgebraicSystem} a set $X=\lbrace x_i\rbrace$ indexed by $i\in I$.  $A$ is said to be a \emph{free algebra} over $\mathcal{K}$, with \emph{free generating set} $X$, if for any algebra $B\in \mathcal{K}$ with any subset $\lbrace y_i\mid i\in I\rbrace \subseteq B$, there is a homomorphism $\phi:A\to B$ such that $\phi(x_i)=y_i$.

If we define $f:I\to A$ to be $f(i)=x_i$ and $g:I\to B$ to be $g(i)=y_i$, then freeness of $A$ means the existence of $\phi:A\to B$ such that $\phi\circ f=g$.

Note that $\phi$ above is necessarily unique, since $\lbrace x_i\rbrace$ generates $A$.  For any $n$-ary polynomial $p$ over $A$, any $z_1,\ldots,z_n \in \lbrace x_i\mid i\in I\rbrace$, $\phi(p(z_1,\ldots,z_n))=p(\phi(z_1),\ldots,\phi(z_n))$.

For example, any free group is a free algebra in the class of groups.  In general, however, free algebras do not always exist in an arbitrary class of algebras.

\textbf{Remarks}.  
\begin{itemize}
\item
$A$ is free over itself (meaning $\mathcal{K}$ consists of $A$ only) iff $A$ is free over some equational class.
\item
If $\mathcal{K}$ is an equational class, then free algebras exist in $\mathcal{K}$.
\item
Any term algebra of a given structure $\tau$ over some set $X$ of variables is a free algebra with free generating set $X$.
\end{itemize}
%%%%%
%%%%%
\end{document}
