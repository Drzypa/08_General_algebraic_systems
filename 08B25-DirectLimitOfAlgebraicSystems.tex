\documentclass[12pt]{article}
\usepackage{pmmeta}
\pmcanonicalname{DirectLimitOfAlgebraicSystems}
\pmcreated{2013-03-22 16:53:56}
\pmmodified{2013-03-22 16:53:56}
\pmowner{CWoo}{3771}
\pmmodifier{CWoo}{3771}
\pmtitle{direct limit of algebraic systems}
\pmrecord{7}{39156}
\pmprivacy{1}
\pmauthor{CWoo}{3771}
\pmtype{Definition}
\pmcomment{trigger rebuild}
\pmclassification{msc}{08B25}
\pmsynonym{direct system of algebraic systems}{DirectLimitOfAlgebraicSystems}
\pmsynonym{inverse system of algebraic systems}{DirectLimitOfAlgebraicSystems}
\pmsynonym{projective system of algebraic systems}{DirectLimitOfAlgebraicSystems}
\pmrelated{DirectLimitOfSets}
\pmdefines{direct family of algebraic systems}
\pmdefines{inverse family of algebraic systems}
\pmdefines{inverse limit of algebraic systems}

\usepackage{amssymb,amscd}
\usepackage{amsmath}
\usepackage{amsfonts}

% used for TeXing text within eps files
%\usepackage{psfrag}
% need this for including graphics (\includegraphics)
%\usepackage{graphicx}
% for neatly defining theorems and propositions
\usepackage{amsthm}
% making logically defined graphics
%%\usepackage{xypic}
\usepackage{pst-plot}
\usepackage{psfrag}

% define commands here
\newtheorem{prop}{Proposition}
\newtheorem{thm}{Theorem}
\newtheorem{ex}{Example}
\newcommand{\real}{\mathbb{R}}
\begin{document}
An immediate generalization of the concept of the direct limit of a direct family of sets is the direct limit of a direct family of algebraic systems.

\subsubsection*{Direct Family of Algebraic Systems}

The definition is almost identical to that of a direct family of sets, except that functions $\phi_{ij}$ are now homomorphisms.  For completeness, we will spell out the definition in its entirety.

Let $\mathcal{A}=\lbrace A_i\mid i\in I\rbrace$ be a family of algebraic systems of the same type (say, they are all $O$-algebras), indexed by a non-empty set $I$.  $\mathcal{A}$ is said to be a \emph{direct family} if 
\begin{enumerate}
\item $I$ is a directed set,
\item whenever $i\le j$ in $I$, there is a homomorphism $\phi_{ij}:A_i\to A_j$,
\item $\phi_{ii}$ is the identity on $A_i$,
\item if $i\le j\le k$, then $\phi_{jk}\circ \phi_{ij}=\phi_{ik}$.
\end{enumerate}

An example of this is a direct family of sets.  A homomorphism between two sets is just a function between the sets.

\subsubsection*{Direct Limit of Algebraic Systems}

Let $\mathcal{A}$ be a direct family of algebraic systems $A_i$, indexed by $I$ ($i\in I$).  Take the disjoint union of the underlying sets of each algebraic system, and call it $A$.  Next, a binary relation $\sim$ is defined on $A$ as 
follows: 
\begin{quote}
given that $a\in A_i$ and $b\in A_j$, $a\sim b$ iff there is $A_k$ such that $\phi_{ik}(a)=\phi_{jk}(b)$.
\end{quote}

It is shown \PMlinkname{here}{DirectLimitOfSets} that $\sim$ is an equivalence relation on $A$, so we can take the quotient $A/\sim$, and denote it by $A_{\infty}$.  Elements of $A_{\infty}$ are denoted by $[a]_I$ or $[a]$ when there is no confusion, where $a\in A$.  So $A_{\infty}$ is just the direct limit of $A_i$ \emph{considered as sets}.

Next, we want to turn $A_{\infty}$ into an $O$-algebra.  Corresponding to each set of $n$-ary operations $\omega_i$ defined on $A_i$ for all $i\in I$, we define an $n$-ary operation $\omega$ on $A_{\infty}$ as follows:
\begin{quote}
for $i=1,\ldots,n$, pick $a_i\in A_{j(i)}$, $j(i)\in I$.  Let $J:=\lbrace j(i)\mid i=1,\ldots,n\rbrace$.  Since $I$ is directed and $J$ is finite, $J$ has an upper bound $j\in I$.  Let $\alpha_i=\phi_{j(i)j}(a_i)$.  Define
$$\omega([a_1],\ldots,[a_n]):=[\omega_j(\alpha_1,\ldots,\alpha_n)].$$
\end{quote}

\begin{prop} $\omega$ is a well-defined $n$-ary operation on $A_{\infty}$. \end{prop}
\begin{proof}  Suppose $[b_1]=[a_1],\ldots, [b_n]=[a_n]$.  Let $\alpha_i$ be defined as above, and let $a:=\omega_j(\alpha_1,\ldots,\alpha_n)\in A_j$.  Similarly, $\beta_i$ are defined: $\beta_i:=\phi_{k(i)k}(b_i)\in A_k$, where $b_i\in A_{k(i)}$.  Let $b:=\omega_k(\beta_1,\ldots,\beta_n)\in A_k$.  We want to show that $a\sim b$.  

Since $a_i\sim b_i$, $\alpha_i\sim \beta_i$.  So there is $c_i:= \phi_{j\ell(i)}(\alpha_i)=\phi_{k\ell(i)}(\beta_i)\in A_{\ell(i)}$.  Let $\ell$ be the upper bound of the set $\lbrace \ell(1),\ldots,\ell(n)\rbrace$ and define $\gamma_i:=\phi_{\ell(i)\ell}(c_i)\in A_{\ell}$.  Then 
\begin{eqnarray*}
\phi_{j\ell}(a)&=& 
\phi_{j\ell}\big(\omega_j(\alpha_1,\ldots,\alpha_n)\big) \\ &=& 
\omega_{\ell}\big(\phi_{j\ell}(\alpha_1),\ldots,\phi_{j\ell}(\alpha_n)\big) \\ &=& \omega_{\ell}\big(\phi_{\ell(1)\ell}\circ \phi_{j\ell(1)}(\alpha_1),\ldots, \phi_{\ell(n)\ell}\circ \phi_{j\ell(n)}(\alpha_n)\big) \\ &=& 
\omega_{\ell}\big(\phi_{\ell(1)\ell}(c_1),\ldots,\phi_{\ell(n)\ell}(c_n)\big) \\ &=& \omega_{\ell}\big(\phi_{\ell(1)\ell}\circ \phi_{k\ell(1)}(\beta_1),\ldots, \phi_{\ell(n)\ell}\circ \phi_{k\ell(n)}(\beta_n)\big) \\ &=& 
\omega_{\ell}\big(\phi_{k\ell}(\beta_1),\ldots,\phi_{k\ell}(\beta_n)\big) \\ &=& \phi_{k\ell}\big(\omega_k(\beta_1,\ldots,\beta_n)\big) \\ &=& \phi_{k\ell}(b),
\end{eqnarray*}
which shows that $a\sim b$.
\end{proof}

\textbf{Definition}.  Let $\mathcal{A}$ be a direct family of algebraic systems of the same type (say $O$) indexed by $I$.  The $O$-algebra $A_{\infty}$ constructed above is called the \emph{direct limit} of $\mathcal{A}$.  $A_{\infty}$ is alternatively written $\varinjlim A_i$.

\textbf{Remark}.  Dually, one can define an \emph{inverse family of algebraic systems}, and its inverse limit.  The inverse limit of an inverse family $\mathcal{A}$ is written $A^{\infty}$ or $\varprojlim A_i$.
%%%%%
%%%%%
\end{document}
